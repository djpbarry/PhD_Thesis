\section{Materials \& methods}

\subsection{Micro-organism cultivation}

The basal medium (BM; see Section~\ref{sec:BasalMedium}) used for all fermentations was that described for batch cultivation of \emph{A. oryzae} by Amanullah and colleagues \cite{amanullah2000}, with soluble starch (Sigma S-9765, Lot 93H0243 or Difco 0178-17-7, Lot FJ0041XA; concentration specified in individual experiments) used in place of maltodextrin and Pluronic P6100 omitted. Cultivations were conducted at pH 6 (reported by Carlsen and colleagues as optimal for maximal specific $\alpha$-amylase production \cite{carlsen1996a}) unless otherwise stated and a temperature of 25 -- \celc{30}. For investigations of non-ionic detergents and polymers, Nonidet P-40, Triton X-100, Tween-80, carboxymethylcellulose (CMC), Ficoll, diethylaminoethyl cellulose (DEAE), Sephadex G-75 and Sephadex G-200 were added to a final concentration of 0.05 -- 1.0\% (w/v). In all shake-flask fermentations, 250~ml Erlenmeyer flasks, with a working volume of 20\%, were incubated in a Lh Fermentation Mk X Incubator Shaker at 200~rpm. Inoculum consisted of \ul{500} of \inoc{1}{7} unless otherwise stated.

Investigations into particle agglomeration were conducted using malt extract broth: malt extract (Difco 0186-17, Lot 138225XD), \gl{17.0}; bacteriological peptone (Oxoid LP0037, Lot 239324), \gl{3.0}. Solid state cultivation was conducted on BM supplemented with Agar No.~1 (\gl{17.0}) and immobilisation of fungal spores was performed as described in Section~\ref{sec:OptAssay}.

\subsubsection{Mixed-phase cultivation}

Cellulose nitrate membranes were overlaid onto the surface of Sabouraud dextrose agar (SDA): Sabouraud liquid medium (Lab M LAB033), \gl{30.0}; Agar No. 1 (Lab M), \gl{17.0}. The membranes were inoculated with 500~$\mu$L of \inoc{1}{7} and incubated at \celc{30} for 16~hours. The membranes were then aseptically removed from the surface of the agar and transferred into Erlenmeyer flasks containing 50~ml of pre-autoclaved medium. The cultivation conditions for the shake-flask phase are referred to in individual experiments.

\subsection{Visualisation of fungal morphology}

Fungal macro-morphology was imaged by either photographing unstained biomass against a black background, or by using the method described in Section~\ref{sec:PelletScan}. The analysis of these images was performed as described in Section~\ref{sec:MacroAnalysis}. Visualisation of fungal micro-morphology and the processing of solid-state-cultured membranes for image analysis was as described in Section~\ref{sec:OptAssay}. Fungal micro-morphology was quantified as described in Section~\ref{sec:MicroAnalysis}. The effect of detergents on particle dispersal in submerged media was evaluated using a variation of the approach adopted by Grimm and colleagues \cite{grimm2004}; samples taken from shake-flasks were processed and imaged as above, then, using the routines developed for the quantification of spore morphology described in Chapter~\ref{ch:DevImagAnal}, the number of particles (any object detected within field of view) per unit volume of media was calculated.

\subsection{Processing of shake-flask cultures}

\subsubsection{Estimation of dry-cell weight}

The dry-cell weight of biomass per unit volume was estimated in one of two ways. In the first method, flask contents were transferred to a clean, dry, pre-weighed plastic universal and centrifuged at 3,000~rpm for 35~minutes at \celc{4}. Any wall-adherent biomass from the original flask was removed by washing with PBS containing Tween-80 (0.1\% v/v). The universal containing biomass was dried in an oven for 24~hours at \celc{105}, then allowed to cool in a desiccator for 30 -- 60~minutes before being re-weighed.

In the second method, flask contents were vacuum-filtered through a glass microfibre filter (Whatman GFC 1822-110). Any wall-adherent biomass from the flask was removed by washing with PBS-T80 (0.1\% w/v). The filter and biomass were transferred to a pre-weighed universal and dried for 24~hours at \celc{105}, then allowed to cool in a desiccator for 30 -- 60~minutes before being re-weighed. The weight of the filter (determined by drying three filters at \celc{105} for 24~h, then weighing) was subtracted from the final result.

\subsubsection{Estimation of $\alpha$-amylase activity and extra-cellular protein concentration}

Thimerosal (Sigma T8784) and protease inhibitor cocktail (Sigma P8340) were added to culture supernatants at final concentrations of 0.01\% w/v and 0.001\% v/v (of stock concentration) respectively. The samples were then stored at \celc{-20} prior to analysis.

$\alpha$-amylase activity was estimated with the use of a ceralpha assay kit (Megazyme K-CERA). The ceralpha procedure employs as substrate the defined oligosaccharide \lq non-reducing-end blocked $p$-nitrophenyl maltoheptaoside' in the presence of excess levels of $\alpha$-glucosidase. On hydrolysis of the oligosaccharide by $\alpha$-amylase, the excess quantities of $\alpha$-glucosidase results in hydrolysis of the $p$-nitrophenyl maltosaccharide fragment to glucose and free $p$-nitrophenol, which may be quantified by measuring absorbance at 400~nm. The assay procedure (including preparation of assay \lq blank') was conducted according to the manufacturer's manual \cite{megazyme}. Absorbance values were measured in plastic cuvettes (Sarstedt 67.742) using a Pharmacia LKB Ultraspec III spectrophotometer blanked against distilled water. Activity was expressed in international  units (IU), derived from ceralpha units (CU) according to the manufacturer's manual:

\begin{equation}
	\mbox{IU} = 0.94 \times \mbox{CU}
\end{equation}

Extra-cellular protein concentration was determined using the Bradford colourimetric protein assay (Bio-Rad 500-0006) with bovine serum albumin (Sigma A7906, Lot 115K0714) as standard \cite{bradford1976}. The assay was conducted in plastic micro-titre plates (Sarstedt 82.1582) and absorbance values measured using a Labsystems Multiskan Plus plate reader.