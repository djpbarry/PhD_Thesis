\section{Discussion}

The yield from fermentation processes involving filamentous micro-organisms are heavily influenced, both directly and indirectly, by the phenotype adopted by the microbe \cite{papagiannireview,znidarsic2001}. While many reports have indicated success in elucidating reproducible relationships between morphology and metabolite production \cite{carlsen1996a,truong2004,elenshasy2006,couri2003,papagianni2006a,xu2000, papagianni1998, jppark2002, dobson2008b, dobson2008a}, conflicting reports do exist  \cite{kisser1980,paul1999,ali2006}, while others have failed to demonstrate any significant dependence \cite{johansen1998,amanullah1999,papagianni1994,muller2002,amanullah2002}. The optimisation of submerged fermentations through the computation of morphological parameters therefore remains a considerable challenge for fungal biotechnologists. While growth on solid culture may be essentially restricted to two dimensions (using membrane-immobilisation), filamentous micro-organisms cultivated in liquid medium typically exhibit significant three-dimensional character, resulting in complex conformations that are not easily quantified. Consequently, \lq free' filaments are generally targeted for detailed microscopic analysis, as individual hyphae and branches may be isolated and an accurate description of morphological parameters produced. However, the presence of such free elements may be rare, depending on the culture conditions. Nevertheless, given the evidence for metabolite excretion occurring primarily at hyphal tips \cite{wosten1991,muller2002}, microscopic characterisation is essential for a thorough description of microbial growth. 

Furthermore, there is evidence that the microscopic development of an organism can have a significant influence on the macroscopic form. For example, a \lq hyper-branched' mutant of \emph{A. oryzae}, which exhibited  a lower projected area per hyphal tip, resulted in more compact clumps and pellets compared to a wild-type strain, resulting in lower broth viscosity \cite{muller2002,muller2003}. Park and colleagues also demonstrated relationships between micro-morphological parameters and macroscopic form in different species of \emph{Mortierella}, although the resultant macro-morphologies were not quantitatively assessed \cite{eypark2006}. Species exhibiting a high branch formation rate formed \lq pellet-like' structures with a distinct core, while species attributed with a low branch formation rate formed hyphal aggregates without cores. Microscopic analysis of early hyphal development is therefore essential if the formation of diverse phenotypes is to be understood.

\subsection{Growth and fragmentation of pellets in shake-flask culture}

The characterisation of shake-flask culture conducted here indicated that the distribution of pellet sizes at any given point in time is dependent on both growth and fragmentation. The depth of the \lq active' region of \emph{A. oryzae} pellets has been estimated to be of the order of \mic{100 -- 300} \cite{carlsen1996a}, meaning that by 24~hours post-inoculation, when pellets had attained a mean diameter of approximately 1.5~mm (Fig.~\ref{fig:BasicFermPellets}), oxygen limitation had almost certainly set in at the pellet core. Furthermore, considering that Carlsen and colleagues demonstrated a sharp increase in ethanol concentration (indicating anaerobic metabolism) at the onset of pellet fragmentation in \emph{A. oryzae} just 35~hours post-inoculation \cite{carlsen1996a}, it seems likely that in the shake-flask culturing conducted here, anaerobic conditions existed at the pellet core by the time fragmentation was observed (when $t \geq 72$~h). Autolysis and subsequent \lq hollowing' of pellets may also offer an explanation for the reduction in biomass, which was concurrent with pellet fragmentation.

However, if autolysis was solely responsible for pellet fragmentation, then it might be expected that the fragmented pellets would continue to grow and biomass would continue to increase. This was clearly not the case (Fig.~\ref{fig:BasicFermDp}), suggesting that the culture was nutrient-limited. Under these conditions, vacuolation of hyphae at the pellet periphery may have weakened the structure of these cells \cite{papagianni1999}, resulting in shearing of the pellet boundary and a subsequent reduction in pellet size, while also causing an increase in the number of relatively small mycelial objects in the media (Fig.~\ref{fig:DpDist}). It is possible that pellet fragmentation resulted from a combination of mass-transfer limitations (perhaps leading to autolysis) and nutrient exhaustion.

As discussed in Chapters~\ref{ch:DevImagAnal} and \ref{ch:NitroAssay}, contiguous, high-contrast staining of hyphae is essential for accurate quantification of hyphal architectures using automated image analysis. As such, quantification of pellet/hyphal fragments at the microscopic level was not possible due to limited stain uptake by hyphae (Fig.~\ref{fig:LPCBWetStain}). Indeed, comprehensive assessment of micro-morphology in batch culture over a period of days (rather than hours) are rare in the literature. One such study was conducted by M\"{u}ller and colleagues, who characterised the microscopic development of \emph{A. oryzae} up to 60~hours post-inoculation \cite{muller2003}. However, a relatively low resolution was employed ($\times 4$ objective) and manual counts of hyphal tips were required. A higher resolution examination was conducted by Carlsen and colleagues, but was restricted to the first 18 hours of growth \cite{carlsen1996a}. However, examination of microscopic structures that result from pellet/hyphal fragmentation may be of limited value as such objects will exhibit artificial \lq tips' resulting from hyphal fragmentation. While it may be of interest to identify points on hyphae at which breakage has occurred, discriminating between \lq true' tips and artificial tips using bright-field microscopy would be difficult to achieve. However, an analysis of such fragments at the microscopic level could be used to provide more complete data on the projected area of biomass elements than that which was presented here.

More extensive morphological data could also be gathered through the microscopic examination of pellets (such as that presented in Table~\ref{tab:TweenPelletMorph}). Pellet diameter and circularity were quantified by Paul and colleagues by suspending pellets of \emph{A. niger} in water (to preserve the three-dimensional structure) in a Petri dish and imaging with a macro-viewer \cite{paul1999}. However, given the problems associated with hyphal staining encountered here, it is likely that stain uptake around the pellet periphery would be poor in the latter stages of the fermentation. Furthermore, the pellets encountered in this work were often far too large to microscopically image in a single field of view, with pellet diameters of up to $\sim 7$~mm measured at low inoculum concentrations.

\subsection{Limitations of mixed-phase culture format}

Investigations into the feasibility of a mixed-phase culture format were limited by a lack of morphological data; there was little potential for analysis of the macroscopic form adopted by the organism (Fig.~\ref{fig:MixedMorpha} to \ref{fig:MixedMorphd}). While microscopic examination of membrane-immobilised cultures is possible (Chapters~\ref{ch:DevImagAnal} \& \ref{ch:NitroAssay}), the situation is complicated somewhat when a liquid medium is employed, as the fungus must be allowed to establish itself on the membrane (for the purpose of adherence). This establishment prior to submerged incubation limits the \lq window' during which microscopic visualisation may be conducted, although a \lq snapshot' at a particular point in time is still possible. This difficulty may potentially be overcome by employing some form of \lq adhesive' to fix spores to the membrane, such as poly-{\scshape d}-lysine \cite{spohr1998}, obviating the requirement for pre-establishment. This may permit a detailed assessment of early hyphal development in the mixed-phase system, although based on the results presented here, it seems likely that the resultant growth would exhibit a significant three-dimensional character.

There are reports in the literature of successful attempts to incorporate solid supports into liquid fermentation systems. While no significant influence on metabolite production was found in this investigation, the use of polymeric membranes in liquid culture has recently been discovered to result in elevated production in cultures of \emph{Penicillium} sp. \emph{LL}-WF159 \cite{bigelis2006}. In addition, nylon supports were found to substantially increase citric acid yield when employed in submerged cultures of \emph{A. niger} \cite{papagianni2004}. The increased production was attributed to the considerably lower diffusion path in the immobilised mycelium compared to pelleted cultures. However, morphological characterisation of immobilised biomass was limited to qualitative assessment of images produced using scanning electron microscopy.

\subsection{Macro-morphological influence of starch concentration}

A significant macroscopic response was observed when the concentration of starch in the media was increased (Fig.~\ref{fig:VaryCPelletMorpha} to \ref{fig:VaryCPelletMorphd}). While it was not possible to investigate this influence at the microscopic level (due to high levels of artifact resulting from suspended starch), evidence in the literature suggests that increasing substrate concentration serves only to increase growth rate, without affecting micro-morphological parameters. For example, Park and colleagues reported that the branch formation rate of \emph{Mortierella alpine} was independent of carbon concentration \cite{eypark2002}. A similar conclusion was reached by Spohr and colleagues in the study of \emph{A. oryzae}, where Monod kinetics were used to illustrate the effect of substrate concentration on growth parameters \cite{spohr1998}. However, M\"{u}ller and colleagues reported that, in \emph{A. niger} and \emph{A. oryzae}, the length of the apical compartment, the number of nuclei in the apical compartment and the hyphal diameter were regulated in response to the surrounding glucose concentration \cite{muller2000}; variations in length of approximately \mic{100} were recorded for different growth rates. However, the concentrations of starch used in this study were likely to be high enough to ensure that $\mu$ approximated $\mu_{max}$. Hence, variations in apical volume caused by changes in specific growth rate seem unlikely.

The provision of additional substrate resulted in an approximately linear increase in biomass levels (Fig.~\ref{fig:AaDCWS}); a similar linear increase in biomass for increasing starch concentration was reported by Nahas and Waldemarin in the culturing of \emph{A. ochraceus} \cite{nahas2002}. However, it is clear that not only did an increase in starch concentration result in additional biomass, but a significant influence on morphology was also observed, which may offer an explanation for the lower yields of $\alpha$-amylase per DCW. The growth form that resulted at high starch concentrations (Fig.~\ref{fig:VaryCPelletMorphd}) produced a much more turbid medium compared to that which prevailed at lower substrate concentrations (Fig.~\ref{fig:VaryCPelletMorpha}). The predominance of this \lq pulpy' growth may have caused mixing problems in the broth, resulting in mass transfer limitations and a subsequent reduction in metabolite production.

A macro-morphological influence of substrate concentration has been reported in other studies. For example, an increase in the pellet size of \emph{Mortierella alpina} was described for increasing carbon to nitrogen ratios (for $C/N > 20$), but when the medium was enriched at a fixed $C/N$ ratio of 20, the whole pellet size and the width of the pellet annular region decreased with increasing nutrient concentration \cite{koike2001}. Papagianni and Mattey found that mean equivalent pellet diameter was inversely related to glucose concentration in shake-flask fermentations of \emph{A. niger} \cite{papagianni2004}.

\subsection{Impact of carbon source variation}

A limited macro-morphological influence of carbon source variation was observed (Fig.~\ref{fig:AaDCWApC}), with the mean equivalent diameter of pellets cultivated in the presence of starch being slightly less than that in the presence of maltose (approximately 10\%) and glucose (15\%). It was also noted that biomass yield was 15 -- 20\% lower on starch compared to maltose and glucose, which may suggest that the differences in pellet size were linked to variations in the extent of mycelial growth. Although differences in both pellet size and $\alpha$-amylase production were discernible, relating both of these parameters is not feasible in this case given the physiological implications of varying carbon sources. A significant micro-morphological effect of carbon source variation was not evident, although the mycelia analysed were relatively small and unbranched (Table~\ref{tab:VaryCMicroMorph}).

The high yield of $\alpha$-amylase on starch compared to the low yield obtained on glucose (Fig.~\ref{fig:AaDCWApC}) is consistent with other investigations into the effects of different substrates. Agger and colleagues found that starch was the best inducer of $\alpha$-amylase production in both batch and continuous cultivation of different strains of \emph{A. nidulans}, compared with glucose, maltose and various mixtures of carbon sources \cite{agger2002}, while glucose has been shown to significantly repress $\alpha$-amylase production in \emph{A. oryzae} \cite{carlsen1996b}. Repression of $\alpha$-amylase production by glucose was also reported by Nahas and Waldemarin, who suggested lactose, maltose, xylose and starch as suitable substrates for inducing $\alpha$-amylase production in stationary culturing of \emph{A.~ochraceus} \cite{nahas2002}.

\subsection{Influence of inoculum concentration on morphology and metabolite production}

The influence of inoculum concentration on both macro-morphology and $\alpha$-amylase production was significant (Fig.~\ref{fig:Inoc}); a high inoculum concentration was preferable for the formation of small pellets and, consequently, higher $\alpha$-amylase activity. It may be the case that a further increase in inoculum concentration beyond \inoc{1}{8} leads to further reductions in pellet size and concomitant increases in $\alpha$-amylase yield, but whether filamentous growth (which has been found to be preferable for $\alpha$-amylase production from \emph{A. oryzae} \cite{carlsen1996a}) may be induced in such a manner is unclear. Bizukojc and Ledakowicz proposed a linear relationship between inoculum concentration ($C_i$) and the number of pellets per unit volume ($n_{pellets}$) that resulted in cultures of \emph{A. terreus} \cite{bizukojc2009}:

\begin{equation}
	n_{pellets} = a . C_i
\end{equation}

\noindent It was proposed that a constant number of spores ($a \approx 10,400$) formed the core of each pellet. This indicated that while an increase in inoculum concentration would result in a decrease in pellet size, this decrease was owing to a reduction in available nutrients per pellet (and consequently, a reduction in the extent of growth) rather than increased spore dispersal. This suggests that increasing the inoculum concentration in cultures of \emph{A. terreus} is unlikely to induce filamentous growth and the same may also be true of \emph{A. oryzae}, given that pellets are formed by this organism through a similar agglomerative process \cite{carlsen1996a}. Furthermore, if the number of spores forming each pellet is constant, then the mean amount of biomass ($X$) produced by each pellet is equal to the available nutrient concentration ($S$) divided equally among all pellets:

\begin{equation}
	X = SY_{X/S} \ . \ \frac{1}{n_{pellets}}
\end{equation}

\noindent where $Y_{X/S}$ is the yield of biomass per unit volume of media. Taking the projected area of pellets ($\Ap$) to be proportional to biomass ($\Ap=b.X$, $b$ is a constant) and substituting for $n_{pellets}$:

\begin{equation}
	\Ap = \frac{bSY_{X/S}}{a} \ . \ \frac{1}{C_i}
\end{equation}

\noindent This is similar in form to the relationship between $\Dp$ and $C_i$ shown in Figure~\ref{fig:AmylasePelletInoc}.

What is striking is the sensitivity of the system to relatively small changes in the initial spore concentration, emphasising the need for strict maintenance of inoculum suspensions and the accurate evaluation of spore viability. A large reduction in the diameter of \emph{A. oryzae} pellets for an increase in inoculum concentration of approximately one order of magnitude was also reported by Truong and colleagues, without having an appreciable effect on the \lq hairiness' of the pellets (ratio of area of outer filamentous zone to that of the core) \cite{truong2004}. However, the cultivations were conducted in cassava starch processing waste-water, containing suspended solids, which were shown to influence pellet formation. A similar range of variation in inoculum concentration was also investigated by Xu and colleagues in the fermentation of \emph{A. niger} and while the observed differences in pellet size were significant, the variation was approximately linear \cite{xu2000}.

It should also be noted that a micro-morphological influence of variation in inoculum concentration cannot be discounted, as it has been demonstrated that increasing spore concentration causes an increase in the hyphal growth unit of \emph{A. niger} \cite{papagianni2006b} and \emph{A. awamori} \cite{johansen1998}, which may have implications for the relationship proposed in Equation~\ref{eq:aaNA}. However, the specific rate of $\alpha$-amylase production (\h{IU~mg\sp{-1}~DCW}) has been reported as being closely coupled to the growth of \emph{A. oryzae} \cite{carlsen1996b}. Furthermore, it was suggested by Carlsen and colleagues that the product of the specific hyphal branching rate ($\kb$) and the maximal hyphal tip extension rate ($\kt$) may be related to the square of the specific growth rate of freely dispersed cultures \cite{carlsen1996a}:

\begin{equation}
	\mu^2 = k_{tip}.k_{b}
\end{equation}

\noindent The branching kinetics of an organism can be reasonably described using $\kt$ and $\kb$ and, therefore, if specific $\alpha$-amylase production is proportional to $\mu$ and $\mu$ is independent of the inoculum concentration, then Equation~\ref{eq:asDp} should hold true, independent of variations in branching rates. However, in the case of pelleted growth, specific amylase production would be expected to decrease as the fermentation proceeds and more biomass becomes diffusion limited, which would also result in a reduction in the specific growth rate. It must therefore be assumed that both $\mu$ and the specific amylase production rate are dependent on $\Ci$ and microscopic analysis is required to confirm the relationship between morphology and metabolite production.

\subsection{Supplementation of cultures with surfactant compounds}

The lack of a reduction in pellet size in fermentations supplemented with Tween-80 (Fig.~\ref{fig:DpT80}) is in disagreement with evidence in the literature suggesting that the inclusion of surfactant compounds leads to smaller pellets and/or dispersed growth. The presence of Tween-80 was found to have a dispersive effect in the fermentation of \emph{T. reesei}, with a \lq pulpy' growth form resulting from its inclusion \cite{domingues2000}. However, a slight increase in the size of \emph{R. nigricans} pellets was observed by \v{Z}nidar\v{s}i\v{c} and colleagues when cultures were supplemented with Tween-80, although higher concentrations also resulted in some dispersed growth \cite{znidarsic2000}.

The increase in the number of suspended particles in the presence of Tween-80 may indicate that the dispersal of the hydrophobic spores is aided by the inclusion of surfactant compounds (Fig.~\ref{fig:AggT}). However, the resultant increase in pellet size for some concentrations of Tween-80 (Fig.~\ref{fig:DpT80}) may suggest that hydrophobicity is not the primary driver of pellet formation under these conditions. The hydrophobic nature of dormant spores has been  confirmed in investigations using atomic force microscopy, but it has also been demonstrated that the surface properties of spores is altered during the swelling process \cite{vanderaa2002}. Indeed, Dague and colleagues have offered evidence that spores of \emph{A. fumigatus} become hydrophilic as they swell \cite{dague2008}; cell surface hydrophobicity can have a significant influence on morphology in submerged cultures \cite{dobson2008a}. However, Dynesen and Nielsen concluded that both electrical charge and hydrophobicity affects pellet formation of \emph{A. nidulans}, but spore agglomeration cannot be attributed to these factors alone \cite{dynesen2003}.

In cultures supplemented with Nonidet-P40 and Triton X-100, a reduction in pellet size was observed (Fig.~\ref{fig:asDpDet}), although, given the low levels of biomass, this reduction may be owing to inhibited growth of the pellets, rather than a dispersive effect (although some flasks did contain a considerable amount of free mycelia; not shown); Triton X-100 was also reported to inhibit the growth of \emph{S. hygroscopicus} \cite{dobson2008b}. However, all three surfactants yielded increases in $\alpha$-amylase yield, which did not appear to be directly related to pellet size (Fig.~\ref{fig:asDpT80} and \ref{fig:asDpDet}), as was the case in modifying morphology via variations in inoculum concentration. It is possible that the surfactants affected the microscopic development of the organism, but examination of hyphal elements isolated from cultures supplemented with Nonidet P-40 or Triton X-100 was not possible, as the detergents apparently inhibited the staining of the hyphae (not shown). However, quantification of the growth of \emph{A. oryzae} on solid substrate in the presence of Tween-80 would seem to suggest that the surfactants did not have a significant microscopic influence (Fig.~\ref{fig:MicroT80}). However, given the different chemical structures and the inhibitory effects of Triton X-100 and Nonidet-P40, coupled with the inherent environmental differences between the submerged and solid-state systems, an influence at the microscopic level (and possibly a physiological influence) cannot be ruled out.

There also exists the possibility that the presence of non-ionic detergents caused an increase in cell permeability, resulting in greater diffusion into pellets and subsequently raising the proportion of \lq active' biomass. Such a conclusion was reached in studies involving \emph{R. nigricans} \cite{znidarsic2000} and \emph{T. reesei} \cite{domingues2000}, in which cultures were supplemented with surfactants and an increase in metabolite production or biomass yield (or both) was recognised. Correlations between the level of active biomass in a culture and metabolite production have been demonstrated for \emph{A. niger} \cite{elenshasy2006} and \emph{S. fradiae}  \cite{ypark1997}. Alternatively, the increase in $\alpha$-amylase activity observed in the presence of non-ionic detergents may have been a result of enhanced activation of the enzyme by the surfactants, rather than an increase in enzyme production by \emph{A. oryzae}. Yoon and Robyt found that supplementing a solution containing porcine pancreatic $\alpha$-amylase with 0.02\% w/v Triton X-100 resulted in an increase in activity of approximately 40\% \cite{yoon2005}. Furthermore, the activity of Triton-supplemented solutions remained stable over time, while the activity of a control solution (without surfactant supplementation) declined rapidly.

\section{Conclusions}

Fungal morphology is a critically important fermentation parameter, impacting metabolite production and environmental conditions both directly and indirectly. While the relationship between growth form and metabolite yield is sometimes ambiguous, there is a substantial body of evidence suggesting that many processes may be optimised through the selection of environmental variables that favour one particular growth form over others. The original aim of the experimentation described in this chapter was to translate the successful quantification of microscopic development on solid substrates (Chapter~\ref{ch:KinSolidSub}) into the submerged culture format; characterising the morphology of \emph{A. oryzae} in a basal system, then by effecting perturbations, influencing variation at the microscopic level that may provide an insight into macroscopic structure formation and $\alpha$-amylase production. While variations in pellet formation were observed in response to changes in certain parameters (varying substrate concentration, inoculum concentration, supplementation with surfactants), microscopic characterisation of branching behaviour was complicated by a variety of factors (presence of solid particles, spore agglomeration, poor stain uptake). However, while the original aim of relating microscopic architecture to macroscopic form was ultimately unattained, several significant findings were reported.

Cultivation of \emph{A. oryzae} in a basal submerged format resulted in a relatively homogeneous population of pellets, the sizes of which were approximately normally distributed prior to the onset of fragmentation. An exhaustion of nutrients seems to be the likely cause of pellet break-up, although oxygen limitations within the pellets and the onset of autolysis may also have been responsible. Investigations involving different carbon sources support the consensus that starch is the preferred substrate for inducement of $\alpha$-amylase activity in \emph{A. oryzae}, while the presence of glucose is associated with low levels of expression. However, increasing starch concentration above approximately 1.0\% (w/v) seemed to have a negative impact on $\alpha$-amylase activity per unit dry-cell weight, possibly due to limited mixing at higher substrate concentrations. The influence of inoculum concentration demonstrated here is in general agreement with other reports in the literature; higher spore concentrations result in smaller pellets, which are favourable for metabolite production. The supplementation of fermentation media with surfactant compounds was favourable for increased $\alpha$-amylase activity per unit dry cell weight, although lower biomass levels were recorded in the presence of Triton X-100 and Nonidet P-40. The reason for this increase in $\alpha$-amylase expression is not fully understood, although an effect on micro-morphological development cannot be ruled out; a physiological influence is also possible. No significant variations in microscopic parameters were observed when cultivations were conducted on a solid substrate in the presence of Tween-80, although the possibility that surfactant compounds effected micro-morphological variation cannot be completely discounted.

Accurate morphological quantification often depends on the presence of free mycelial elements for characterisation of the branching behaviour of the organism. However, many organisms, such as \emph{Aspergillus oryzae}, typically grow in the pelleted form, as the spores of this fungus agglomerate as they swell when cultivated in submerged medium; some means of disrupting this mechanism is therefore required in order to analyse microscopic architectures. A lowering of the culture pH has been utilised in other studies as a means of achieving this, but an acidic medium has also been shown to adversely affect both growth and product formation. Furthermore, the pelleted growth form has been demonstrated as optimal for many processes and is therefore an industrially-relevant phenotype, demanding of attention in itself. Some form of universal measure that may be applied to all growth forms would obviate the requirement for the presence of \lq free' mycelia in the culture and allow direct comparison of different phenotypes, without the need for classification of these structures.