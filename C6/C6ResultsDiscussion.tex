\section{Results \& discussion}

An analysis of the development of \emph{A. oryzae} on malt agar showed that both $D$ and $\hgu$ increased over time and both tended towards approximately constant values (Fig.~\ref{fig:LhguDT}). This suggests that the value of $\hgu$ specific to \emph{A. oryzae} under these growth conditions is reflected in the fractal dimension of the mycelia. The fractal dimension of \emph{Ashbya gossypii} and \emph{Streptomyces griseus} were also found to increase with time during the colonisation of solid substrates \cite{obert1990}. \emph{A. oryzae} and \emph{P. chrysogenum} were grown under a variety of different conditions (Table~\ref{tab:pops}), producing mycelia of varying size and dimension that were quantified in the same manner. The resultant mean values of $D$ obtained for each population were plotted against the mean values of $\hgu$ to yield an approximately logarithmic relationship (Fig.~\ref{fig:DLhgu}):

\begin{equation} \label{eq:DaLhguB}
	D = a \log \hgu +b
\end{equation}

\noindent where $a$ and $b$ are constants. This result demonstrates a strong correlation between the branching behaviour of mycelia and their space-filling properties. However, it has been shown in other studies that fractal dimension tends to increase as projected area of mycelial structures increases \cite{papagianni2006b}. This may also be the case in this study, as higher values of $D$ tended to be biased toward high values of $\Ap$ (Fig.~\ref{fig:DAp}), but this result is inconclusive, as the sizes of mycelia analysed fell within a comparatively small range.

\begin{table}[htbp]
	\centering
	\footnotesize
	\begin{threeparttable}
	\caption{Populations of filamentous fungi analysed, where $n$ is the number of mycelia in each population and the total hyphal length ($\Lh$) is presented as mean $\pm 95$\% confidence interval}
	\label{tab:pops}
	\centering
		\begin{tabularx}{(\textwidth - 1cm)}{l l R l r}
			\toprule
			Organism & $t$ (h) & Substrate & $n$ & $\Lh$ (\omic) \\ \midrule
			\emph{A. oryzae} & 14.0 & Malt Agar\tnote{1} & 86 & $85 \pm 16$\\
			\hspace{0.8cm}$\vdots$ & 15.4 & \hspace{0.8cm}$\vdots$ & 87 & $117 \pm 21$\\
			\hspace{0.8cm}$\vdots$ & 16.9 & \hspace{0.8cm}$\vdots$ & 82 & $169 \pm 41$\\
			\hspace{0.8cm}$\vdots$ & 18.3 & \hspace{0.8cm}$\vdots$ & 78 & $341 \pm 71$\\
			\hspace{0.8cm}$\vdots$ & 19.7 & \hspace{0.8cm}$\vdots$ & 83 & $305 \pm 66$\\
			\hspace{0.8cm}$\vdots$ & 21.1 & \hspace{0.8cm}$\vdots$ & 67 & $428 \pm 98$\\
			\hspace{0.8cm}$\vdots$ & 22.6 & \hspace{0.8cm}$\vdots$ & 59 & $336 \pm 101$\\
			\hspace{0.8cm}$\vdots$ & 24.0 & \hspace{0.8cm}$\vdots$ & 76 & $397 \pm 91$\\
			\hspace{0.8cm}$\vdots$ & 18.0 & Solid medium, Nonidet~P-40 5.0\%~(w/v) & 30 & $190 \pm 44$\\
			\hspace{0.8cm}$\vdots$ & 24.0 & Solid medium, Nonidet~P-40 5.0\%~(w/v) & 26 & $654 \pm 177$\\
			\hspace{0.8cm}$\vdots$ & 15.0 & Solid medium, Triton~X-100 0.1\%~(w/v) & 44 & $177 \pm 39$\\
			\hspace{0.8cm}$\vdots$ & 18.0 & Solid medium, Triton~X-100 0.1\%~(w/v) & 35 & $269 \pm 95$\\
			\hspace{0.8cm}$\vdots$ & 46.0 & Liquid Medium & 38 & $874 \pm 198$\\
			\hspace{0.8cm}$\vdots$ & 72.5 & \hspace{0.8cm}$\vdots$ & 56 & $671 \pm 116$\\
			\hspace{0.8cm}$\vdots$ & 94.0 & \hspace{0.8cm}$\vdots$ & 44 & $527 \pm 126$\\
			\emph{P. chrysogenum} & 20.0 & Malt Agar & 69 & $116 \pm 18$\\
			\hspace{0.8cm}$\vdots$ & 27.0 & Malt Agar & 46 & $378 \pm 73$\\
			\hspace{0.8cm}$\vdots$ & 20.0 & Malt Agar, CaCl\sb{2}.2H\sb{2}O 0.08\%~(w/v) & 68 & $131 \pm 20$\\
			\hspace{0.8cm}$\vdots$ & 27.0 & Malt Agar, CaCl\sb{2}.2H\sb{2}O 0.08\%~(w/v) & 32 & $296 \pm 52$\\
			\hspace{0.8cm}$\vdots$ & 27.0 & Malt Agar, FeCl\sb{2}.4H\sb{2}O 0.11\%~(w/v) & 58 & $140 \pm 26$\\
			\hspace{0.8cm}$\vdots$ & 20.0 & Orange & 39 & $180 \pm 36$\\
			\hspace{0.8cm}$\vdots$ & 19.0 & Rice & 114 & $255 \pm 31$\\
			\hspace{0.8cm}$\vdots$ & 20.0 & Rice \& Bulgar Wheat & 42 & $400 \pm 121$\\ \bottomrule
		\end{tabularx}
		\begin{tablenotes}
		\item [1] Results for \emph{A. oryzae} on malt agar were produced using the images generated in Chapter~\ref{ch:KinSolidSub}.
		\end{tablenotes}		\end{threeparttable}
\end{table}

\begin{figure}[t]
	\centering
	\pstool[width=10.5cm]{../C6/LhguDT}{
		\psfrag{H}[Bc]{\al $\hgu$ (\omic)}
		\psfrag{D}[Bc]{\al $D$}
		\psfrag{T}[Bc]{\al \hspace{2mm}Time (h)}}
	\caption{Temporal variation in mean hyphal growth unit ($\hgu$; $\bl$) and the mean fractal dimension ($D$; $\bs$) of populations of \emph{A. oryzae} cultivated on malt agar. Error bars represent 95\% confidence intervals. Produced using images generated in Chapter~\ref{ch:KinSolidSub}.}
	\label{fig:LhguDT}
\end{figure}

\begin{figure}[t]
	\centering
	\pstool[width=10.5cm]{../C6/DLhgu}{
		\psfrag{H}[Bc]{\al $\hgu$ (\omic)}
		\psfrag{D}[Bc]{\al $D$}}
	\caption{Relationship between the mean hyphal growth unit ($\hgu$) and the mean fractal dimension ($D$) of populations of \emph{Aspergillus oryzae} ($\bs$) and \emph{Penicillium chrysogenum} ($\bl$) mycelia, grown under a variety of different conditions. A logarithmic relationship of the form $D=a \log \hgu +b$ exists between the two parameters, where $a=0.14$ and $b=0.65$ (---; $R^2=0.95$). Error bars represent 95\% confidence intervals.}
	\label{fig:DLhgu}
\end{figure}

\begin{figure}[htbp]
	\centering
	\pstool[width=10.5cm]{../C6/DAp}{
		\psfrag{D}[Bc]{\al $D$}
		\psfrag{A}[Bc]{\al $\Ap$ (\omics)}}
	\caption{Relationship between mean fractal dimension ($D$) of populations of \emph{Aspergillus oryzae} ($\bs$) and \emph{Penicillium chrysogenum} ($\bl$) mycelia and mean projected area ($\Ap$).}
	\label{fig:DAp}
\end{figure}

This result depicts a clear relationship between the branching behaviour of filamentous organisms and the fractal dimension of the resultant mycelial architectures, further emphasising the potential use of fractal analysis in morphological quantification. An ability to extract information on the branching behaviour of an organism by surveying the shape of the mycelial boundary would be highly advantageous in the study of more complex conformations where measures such as the hyphal growth unit are not readily obtainable. Furthermore, as has been demonstrated in other studies \cite{papagianni2006b,cpark2007,jckim2005,jones1993a}, fractal analysis can be applied regardless of the gross morphological form that results in a particular process, allowing a more thorough compilation of data. However, a more complete examination, including more complex structures, is necessary to validate the universal application of fractal analysis.

It has been previously suggested that the box-counting method of fractal dimension enumeration may not be suitable for the analysis of small, relatively unbranched hyphal structures \cite{papagianni2006b,obert1990}. The accuracy of the box-counting method relies on an object being sufficiently great in size so as to allow a reasonably large variation in $\epsilon$ (approximately one order of magnitude has been proposed \cite{obert1990}). Given a value of approximately \mic{4} for $\epsilon_{min}$ (hyphal width is approximately \mic{2 -- 4}), this indicates a minimum value of approximately \mic{40} for $\epsilon_{max}$ in this study, equating to a minimum object \lq diameter' ($L$) of \mic{160} (assuming $\epsilon_{max} \leq L/4$ \cite{obert1990}). However, mycelia smaller than this dimension were often encountered, particularly in the case of \emph{P. chrysogenum}. Further, the number of evaluations of $N(\epsilon)$ is restricted by the image resolution (approximately \mic{1} per pixel in this study). This can obviously be overcome by increasing the image resolution, but this in turn results in a significant increase in memory usage and processing time.

By enumerating the fractal dimension based on the object boundary, considerations of resolution are obviated to some degree, as the boundary can be represented geometrically as a series of equations, or indeed as a single spline, to be sampled as often as is necessary to provide sufficient signal resolution. However, image resolution is still an important consideration, as low-resolution images may not contain an accurate representation of the object boundary. Consideration must also be given to the means used to locate the boundary. In this study, hyphae were uniformly stained and object segmentation from background was accurately performed by grey-scale thresholding. In cases where staining is non-uniform, thresholding may not be suitable and some form of edge-detection algorithm may be required. 

A similar examination of spectral periodicity was employed by Jones and colleagues in their study of the positional relationship between acid phosphatase intracellular concentration and hyphal cellular differentiation in \emph{Pycnoporus cinnabarinus}. Acid phosphatase was detected histochemically on membrane-bound colonies, which were imaged using a camera fitted with a macro-lens. In the resulting image, a series of one-dimensional \lq walks' were made along random colony radii and the fractal dimension calculated according to the power spectrum of the resultant luminance profile along each \lq walk'. Different concentrations of an organic dye were used to effect substrate induction of the enzyme response, which was shown to be statistically correlated according to a fractal power law \cite{jones1995}.

While numerous studies have been conducted in which fractal analysis is utilised to quantify mycelial morphology, few have attempted to link fractal dimension with conventional morphological parameters. However, links have been established between fractal dimension and productivity in some processes. For example, in the optimisation of \emph{Funalia trogii} fermentations, both fractal dimension and mean pellet area were monitored; while no link was established between the two parameters, it was suggested that a relationship may exist between fractal dimension and decolourisation of reactive black 5 \cite{cpark2007}. A positive correlation was also found between fractal dimension and phenol-oxidase expression by \emph{Pycnoporus cinnabarinus}, with both parameters being regulated by media composition \cite{jones1997}.

Where links between fractal dimension and conventional Euclidean measures of morphology have been made, the dependency is often either ambiguous or qualitative in nature. An approximate correlation ($R^2=0.614$) was found between the convexity (defined as the ratio between convex perimeter and respective perimeter) of \emph{Cupriavidus necator} DSM 545 flocs and $D_{BS}$ \cite{finkler2007}. Fractal dimension was shown to be related to broth rheology in the submerged fermentation of \emph{Cephalosporium acremonium} M25 and a link with other morphological measures, such as the number of arthrospores in the media, was also suggested, but not explicitly demonstrated \cite{jckim2005}. A relationship between hyphal growth unit and fractal dimension of mycelia was previously noted in submerged fermentations of \emph{Aspergillus niger}, but the differences in the recorded values of $\hgu$ were ambiguous \cite{ryoo1999}. Further studies of \emph{A. niger} revealed that medium composition had a significant impact on the fractal dimension, the changes in morphology reflected in variations in the size and compactness of mycelial aggregates \cite{papagianni2006b}. The local fractal dimension (determined by the concentric circles method) within a colony of \emph{Trichoderma viride} was found to increase with branching frequency (occurrence of \lq loops' in the mycelium), although the result was rather qualitative in nature \cite{hitchcock1996}. However, successful attempts have been made in associating fractal dimension with growth kinetics. While colony expansion rates were found to differ between different strains of \emph{Cryphonectria parasitica}, fractal dimension was found to correlate with the expansion rate, independent of strain \cite{golinski2008}.

\section{Conclusion}

The optimisation of fermentation processes involving filamentous microbes requires extensive knowledge of morphological development, as productivity is heavily influenced by the specific phenotypic form adopted. The accurate quantification of morphological variation in vegetative mycelia is therefore of the utmost importance, but the characterisation of complex morphologies represents a significant challenge. The utility of conventional measures employed in the analysis of these microbes (such as projected area, perimeter length and circularity) is limited, as they reveal little about the extent of branching of the organism, which is known to be related to metabolite production.

An alternative approach to morphological quantification employs the use of fractal geometry to characterise the spatial distribution of an organism. The self-similar nature of mycelial structures has been demonstrated in numerous studies and there is clearly significant potential benefit in the application of fractal analysis to filamentous microorganisms. What has been lacking in these studies is a firm link between fractal dimension and conventional morphological parameters, such as the hyphal growth unit. This study indicates a strong correlation between these two parameters in the assessment of \lq free' mycelial elements and further investigation involving a wide range of complex conformations is necessary. Future work will focus on elucidating a universal relationship between fractal dimension, branching behaviour and productivity, independent of the gross morphological form encountered. It is hoped that this work will lead to a rapid and efficient means of morphological quantification for use in industrial processes.