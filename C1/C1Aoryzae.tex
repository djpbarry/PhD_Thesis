\section{Model organism: \emph{Aspergillus oryzae}}

The model organism utilised in this study, \emph{Aspergillus oryzae}, is widely used in industry for the production of a variety of enzymes ($\alpha$-amylase, amyloglucosidase, galactosidases, $\beta$-glucanase, glucoamylase, glucosidases, hemicellulase, invertase, lipase), as well as some penicillins, gluconic acid, kojic acid and malic acid \cite{papagiannireview}. It was considered by Machida to be \lq \emph{one of the most potent secretory producers of proteins among filamentous fungi}' \cite{machida2002}. Historically however, it is associated with the production of foods and beverages such as \emph{sake} (rice wine), \emph{sh\={o}ch\={u}} (spirits), \emph{miso} (soybean paste) and \emph{shoyu} (soy sauce) in Japan, where it is known as \emph{koji} mould. Indeed, its future potential use in such solid-culture formats is still receiving considerable attention \cite{machida2008}. \emph{A.~oryzae} has also shown significant potential as an expression host for heterologous protein production \cite{wang2005}, with further investigations revealing differential protein expression depending on culture format \cite{ tebiesebeke2002,oda2006}. Indeed, \emph{A. oryzae} was directly involved in the birth of the modern fungal enzyme industry, as it was discovered that the fungus produced amylases that could be used to saccharify starch in place of barley malt in the production of spirits \cite{smith1983}. The utility of the organism in the treatment of certain industrial waste products has also been investigated \cite{truong2004}.

A long history of extensive use in food production has resulted in \emph{A. oryzae} attaining GRAS status and it has been the subject of many previous studies, providing information on the morphology and physiology of the organism during submerged cultivation \cite{carlsen1996a,spohr1997,carlsen1996b}, including studies of the dependence of morphological form on agitation intensity \cite{amanullah2002} and the dynamics of mycelial aggregation \cite{amanullah2001}. The microscopic development of the fungus has also been studied in detail \cite{spohr1998,muller2002}, even down to the level of individual hyphal tips \cite{haack2006}. In solid-state cultivation, relationships between respiration, biomass and $\alpha$-amylase production have been illustrated \cite{rahardjo2005a, rahardjo2005b}, while the effect of substrate variation and supplementation on amylase production was also quantified \cite{ramachandran2004}. This bank of knowledge provided a solid basis on which to basis further experimental design.