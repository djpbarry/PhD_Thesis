\section{Conclusions}

Filamentous microbes are industrially important organisms used to produce a wide range of compounds for a variety of applications. While solid-state fermentations involving such micro-organisms have been conducted by humans for perhaps thousands of years, submerged culturing became the dominant industrial format over the course of the 20\sp{th} century, primarily due to reduced space requirements, but also the greater ease with which such processes may be controlled. However, productivity may be significantly influenced, both directly and indirectly, by the complex, three-dimensional phenotypes that are manifest in submerged culture, the specific form of which is a result of a variety of factors. The advent of the digital era provided the necessary tools for researchers to develop image processing systems to quantify these elaborate conformations, but fully-automated systems are still rare and many studies still rely on qualitative morphological descriptions.

\section{Aims of this study}

\begin{itemize}
	\item The principal aim of this study was the design of an automated image analysis system for the rapid and accurate characterisation of fungal morphology.
	\item In parallel with this, it was required that a means of sampling fungal cultures be developed that would present the organism in an essentially two-dimensional format for imaging purposes, whilst preserving the fragile fungal architecture.
	\item Finally, this system was to be applied to the study of a model organism, \emph{A. oryzae}, and relationships between micro-morphology, macro-morphology and metabolite production investigated.
\end{itemize}

\subsection{Thesis overview}

\textbf{Chapter 2} provides details on the most commonly used protocols and techniques used throughout the study.

\textbf{Chapter 3} describes the development of an integrated application, based on the ImageJ platform, that may be used as a means of automatically analysing a two-dimensional representation of filamentous fungal micro- or macro-morphology, generating data on populations of spores, mycelia or pellets. In the design of this system, speed-of-execution was prioritised to provide a system capable of near-real-time analysis.

\textbf{Chapter 4} describes the investigation of a two-dimensional growth assay by immobilising fungal spores on cellulose nitrate membranes, using \emph{A. oryzae} as the model organism. The suitability of the assay for producing samples appropriate for the image analysis system developed in Chapter 3 was examined. Efforts to optimise the assay were undertaken, concentrating specifically on minimising processing time.

In \textbf{Chapter 5} the kinetic development of \emph{A. oryzae} on a solid substrate was analysed using the systems described in chapters 3 and 4 to establish the basic kinetic parameters for this system and investigate whether a change in media composition results in a change in any of these kinetic parameters.

In \textbf{Chapter 6}, morphology and $\alpha$-amylase production in shake-flask cultures of \emph{A. oryzae} was characterised. Attempts were then made to perturb the system, through variation in inoculum concentration, carbon source type and concentration and surfactant supplementation. Resultant changes in macro-morphology were quantified and related to concomitant changes in amylase production. Attempts were made to derive simple correlations between pellet size and $\alpha$-amylase yield.

\textbf{Chapter 7} describes an alternative means of quantifying the branching behaviour of filamentous microorganisms using fractal geometry. The fractal dimension of different populations of mycelia was related to the conventional hyphal growth unit and the potential for future use of fractal geometry in the analysis of fungi is discussed.

\textbf{Chapter 8} outlines the overall conclusions derived from this study and discusses the relevance of the results with regard to the literature. Potential future investigations are also considered.