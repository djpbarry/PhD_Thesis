\section{The fungi}

As saprotrophs, fungi play an important role in nutrient cycling in a variety of ecosystems, degrading organic matter to inorganic molecules, which may subsequently be exploited by other organisms \cite{barea2005}. Although traditionally included in many botany curricula and textbooks, fungi are evolutionarily more closely related to animals \cite{shalchian-tabrizi2008}; green plants are phototrophic autotrophs (producing complex organic compounds from simple inorganic molecules via photosynthesis), whereas fungi, like animals, are chemotrophic heterotrophs (organic compounds, such as glucose, are metabolised to release energy and obtain carbon for growth). The Kingdom Fungi consists solely of hyphal species, or those closely related to hyphal species, that are exclusively absorptive in their mode of nutrition \cite{carlile2001}. However, there is no unique, generally accepted classification system at the higher taxonomic levels and there are frequent name changes at every level, from species up to phylum (or division). For example, the phylum designation of the Zygomycota is controversial and it was not included as a formal taxon in the \lq AFTOL classification' of Fungi \cite{hibbett2007}. Efforts to establish and encourage usage of a unified and more consistent nomenclature are ongoing \cite{celio2006,hibbett2007}.

The major phyla of fungi have been classified mainly based on the morphology of sexual reproductive structures. However, asexual reproduction, via efficient dispersal of spores or spore-containing propagules from specially-adapted structures or through mycelial fragmentation, is common; it maintains clonal populations adapted to a specific niche, and allows more rapid dispersal than sexual reproduction \cite{heitman2006}. Some species may allow mating only between individuals of opposite mating type, while others can mate and sexually reproduce with any other individual or itself. In sexually reproducing fungi, compatible individuals may combine by fusing their hyphae together into an interconnected network; this process, anastomosis, is required for the initiation of the sexual cycle. Sexual reproduction exists in all fungal phyla (with the exception of the Glomeromycota), but differences exist between fungal groups, which have been used to discriminate species by morphological differences in sexual structures and reproductive strategies \cite{guarro1999,taylor2000}.

The Ascomycota (commonly known as the \lq sac fungi') are the largest phylum of Fungi, with over 64,000 species, which may be either single-celled (yeasts), filamentous (hyphal) or both (dimorphic). The defining feature of this fungal group is the \lq ascus', a microscopic sexual structure in which non-motile spores, called ascospores, are formed. However, many species of the Ascomycota are asexual, meaning that they do not have a sexual cycle and thus do not form asci or ascospores. Instead, asexual reproduction occurs through the dispersal of conidia, produced from fruiting bodies termed conidiophores, the morphology of which can vary extensively from species to species. Perhaps the most famous member of this phylum is the mould \emph{Penicillium chrysogenum} (formerly \emph{Penicillium notatum}), which, through the production of the antibiotic penicillin, triggered a revolution in the treatment of bacterial infectious diseases in the 20\sp{th} century. Some ascomycetes (\emph{Penicillium camemberti}, \emph{Penicillium roqueforti} and \emph{A. oryzae}, for example) have been employed for hundreds or even thousands of years in the production of various foods. More recently, ascomycete fungi (the genus \emph{Aspergillus} in particular \cite{wang2005}) have proved to be suitable candidates for heterologous protein expression, enabling large-scale microbial production of therapeutic proteins such as insulin and human growth hormone.

Basidiomycota is one of two large phyla that, together with the Ascomycota, comprise the sub-kingdom Dikarya (often referred to as the \lq higher fungi'). The most conspicuous and familiar Basidiomycota are those that produce mushrooms, on which are sexual reproductive structures called basidia that bear basidiospores, although this phylum also includes some yeasts and asexual species. Among the \lq lower fungi', the traditional division of the Zygomycota (\lq pin' or \lq sugar' moulds) consists of organisms typically found in soil and animal dung, as well as on fruits high in sugar content, such as strawberries. Among the more extensively studied fungi in this (disputed) taxon are the Mucorales, containing genera such as \emph{Mucor}, \emph{Rhizopus} and \emph{Mortierella}. Zygomycota are defined and distinguished from all other fungi by sexual reproduction via zygospores and asexual reproduction by sporangia, within which non-motile, single-celled sporangiospores are produced. Most Zygomycota, unlike the so-called \lq higher fungi', form hyphae that generally lack septae.