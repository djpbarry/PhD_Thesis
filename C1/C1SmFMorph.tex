\section{Influence of process variables on morphology and metabolite production in submerged fermentations}\label{sec:ProcVarMorph}

The diffusional limitations associated with pelleted growth would perhaps suggest that a dispersed mycelial morphology is preferable for increased product yield. However, while a pelleted form typically results in a broth exhibiting Newtonian properties, facilitating easier mixing, dispersed growth often results in non-Newtonian broth behaviour. This results in mixing problems within the bioreactor, often leading to substrate gradients and oxygen limitations, which can require substantial power inputs to overcome. For example, Papagianni and Mattey reported that when dispersed or clumped growth of \emph{Aspergillus niger} predominated, dissolved oxygen levels did not exceed 80\% of saturation, while 100\% saturation was maintained for pelleted cultures \cite{papagianni2006a}. The factors that determine the growth form are many and varied, some of which will now be considered, together with the techniques utilised in morphological quantification.

\subsection{Inoculum concentration}\label{sec:InocConc}

The initial inoculum concentration can have a significant influence on the gross morphology of a process (Table~\ref{tab:InocLit}). A very large concentration of spores provides a large number of growth centres and a very limited amount of growth from each can result in nutrient exhaustion. Conversely, if the initial spore concentration is low, substantial growth may be required before nutrient exhaustion occurs. The distribution of biomass may thus be varied between a large number of small elements and a small number of large elements. However, in practice, the relationship is often considerably more complex and while simple, linear correlations between inoculum size and pellet diameter have been reported for some organisms, more abrupt, step-change phenotypic shifts have been described for others.

\begin{table}[htbp]
	\centering
	\footnotesize
	\caption{Reports in the literature indicating a link between morphology and initial inoculum concentration}
	\label{tab:InocLit}
	\begin{tabularx}{(\textwidth - 1cm)}{l R c}
		\toprule
		Organism & Report & Reference \\ \midrule
		\emph{A. awamori} & Increased $\hgu$ and faster growing hyphal tips at higher inoculum concentrations & \cite{johansen1998}\\
		\emph{A. niger} & Reduction in pellet size for increasing inoculum concentration. Pellets were optimal for minimising protease production while maximising green fluorescent protein yield & \cite{xu2000}\\
		\emph{A. niger} & Increases in inoculum concentration induced dispersed morphology. Increase in $\hgu$ also noted as inoculum concentration was increased & \cite{papagianni2006a}\\
		\emph{A. terreus} & Mean pellet diameter was inversely proportional to initial spore concentration, with pellet count per unit volume directly proportional to inoculum concentration & \cite{bizukojc2009}\\
		\emph{P. chrysogenum} & Transition from cultures containing large, smooth and compact clumps to more dispersed growth as inoculum level increased. Sharp transition observed from $5 \times 10^4$ to \inoc{5}{5}. & \cite{tucker1992a}\\
		\emph{R. chinensis} & A 10-fold increase in spore concentration sufficient to shift morphology from single agglomerate to dispersed growth & \cite{teng2009}\\
		\emph{R. nigricans} & Reduction in pellet size for increasing inoculum concentration. & \cite{znidarsic2000}\\
		\emph{S. hygroscopicus} & Increasing spore concentration from $10^4$ to \sm{$10^5$} caused a slight increase in pellet diameter. Further increase to \sm{$10^6$} resulted in a decrease & \cite{ocleirigh2003}\\
		\bottomrule
	\end{tabularx}
\end{table}

A direct linear relationship between the number of pellets formed per unit volume and the initial spore concentration was proposed by Bizukojc and Ledakowicz for \emph{Aspergillus terreus} cultivations \cite{bizukojc2009}. It was estimated that approximately 10,400 spores formed a single pellet and, therefore, a higher inoculum concentration resulted in a greater number of smaller pellets (diameter $< 1.5$~mm), which were favourable for mevinolinic acid (lovastatin) production. A methyl blue-staining and subsequent cross-sectioning technique was used to visualise intra-structural regions of different metabolic activity within pellets. The fraction of actively growing cells at the pellet periphery was correlated with mevinolinic acid production, while the specific ($+$)-geodin formation rate was found to increase as the size of the active region declined. However, the methodology involved in the preparation of pellets for microtome sectioning was laborious and time-consuming and does not lend itself to high-throughput processing.

A decrease in pellet size for increased inoculum concentration was also obtained by Xu and colleagues in cultures of a recombinant \emph{A. niger} strain \cite{xu2000}. An unusual means of evaluating mean pellet diameter (according to Jianfeng and colleagues \cite{jianfeng1998}) was employed, which involved filtering culture samples through a set of sieves with pore sizes in the range 0 -- 6~mm, resulting in pellets being divided into six fractions. The mean diameter ($\Dp$) of the population was estimated as:

\begin{equation}
	\Dp = \frac{1}{2}{\displaystyle \sum_{i=0}^5 E_i (2i+1)}
\end{equation}

\noindent where $E_i$ is the dry-cell weight of pellets in fraction $i$. A sharp reduction in $\Dp$ from 3.6 to 0.5~mm was observed when the inoculum level was increased from $10^5$ to \sm{$10^7$}, with free mycelia also discernible at high inoculum concentrations. Protease activity increased approximately 3-fold and green fluorescent protein (GFP) production declined approximately 64\% when the morphology shifted from pelleted to free mycelial growth. Biomass yield increased 29\% as pellet size decreased, indicating the influence of mass transfer limitation on cell growth; O\sb{2} limitation within pellets was suggested to be responsible for the reduction in protease activity. However the accuracy of $\Dp$ estimates is questionable and, as no microscopic analysis was conducted, microscopic structural variations cannot be ruled out.

Larger increases in inoculum concentration were demonstrated by Papagianni and Mattey to induce a completely dispersed growth form in \emph{A.~niger} \cite{papagianni2006a}. As the inoculum level was increased from $10^4$ to \sm{$10^9$}, a large decrease in the size and \lq compactness' (ratio of area of hyphae in clump/pellet to total area enclosed by perimeter) of pellets and mycelial clumps was found, accompanied by an increase in \lq roughness' (circularity). Morphological variation was also reported at the microscopic level, with increases in the mean main hyphal length, mean total length and branching frequency of mycelia observed as the inoculum level was increased from $10^4$ to \sm{$10^8$}. Large pellets (approx. 1.0 to 1.2~mm\sp{2}) were associated with maximal glucosamine production and it was suggested that a relationship between dissolved oxygen levels (dO\sb{2}) and ammonium ion uptake, which reflects biosynthesis of glucosamine, may have existed.

Sharper transitions in growth form have been reported in other studies. For example, Teng and colleagues found that a ten-fold increase in initial spore concentration was sufficient to shift the morphology of \emph{Rhizopus chinensis} from a single large agglomerate (described as \lq \emph{fully entangled mycelia}') to dispersed mycelial growth \cite{teng2009}. Furthermore, large agglomerates were found to be optimal for mycelium-bound lipase production compared to other phenotypes. An inoculum of \inoc{3.3}{9} resulted in dispersed mycelia, but large agglomerates were obtained for inocula between $3.3 \times 10^7$ and \inoc{3.3}{8}. A micro-morphological influence was also observed, with the mean hyphal length of individual mycelia being lower when large agglomerates formed, while the number of tips per mycelium was higher. However, the prevalence of free mycelia in cultures containing large agglomerates was not specified and, as such, it is not clear what population size these results were based on.

A micro-morphological influence of variation in inoculum concentration was also noted by Johansen and colleagues in their study of \emph{Aspergillus awamori} \cite{johansen1998}. Different values of total hyphal length ($\Lh$) were obtained for different inoculum concentrations in the early stages of fermentation; a lower inoculum resulted in a lower rate of sugar consumption and a larger value of $\Lh$ per element before the onset of substrate exhaustion. A lower inoculum concentration also resulted in a greater degree of branching and lower tip extension rates, although a limited effect on cutinase production was observed. It was suggested that a constant number of vesicles were supplied per unit length of hypha and therefore a greater number of tips results in a lower number of vesicles supplied to each individual tip and, consequently, a lower tip extension rate. Assuming enzyme production to be coupled to specific growth rate, it was proposed that lower production per tip may have been related to slower growing tips, although \lq artificial tips' resulting from fragmentation may have biased measurements. Furthermore, derived trends were based on a limited number of data points and large standard deviations were evident, possibly owing to the low number of tips (5 -- 7) measured per hyphal element.

Other studies have suggested that the influence of inoculum concentration can vary depending on medium composition. Domingues and colleagues reported that higher protein and cellulose levels were produced using a more concentrated inoculum (attributed to the absence of pellet formation) in cultures of \emph{Trichoderma reesei} \cite{domingues2000}. However, when the media was supplemented with Tween-80, yeast extract and peptone, it was found that a lower inoculum concentration produced higher protein and cellulase yields; a more concentrated inoculum resulted in a higher initial growth rate and nutrient exhaustion in the early stages of fermentation, which may have resulted in lower metabolite yields. However, morphological classification was limited to the description of growth as either pelleted or \lq pulpy' and, as such, the full extent of structural variation that may have occurred (and the effect on production) is not known.

\subsection{Mechanical agitation}\label{sec:agit}

Mechanical agitation is a general requirement in the fermentation of filamentous microbes in order to ensure adequate mixing and aeration. However, the turbulent flows generated, together with the shearing forces of impellers, can have a significant influence on morphology (Table~\ref{tab:AgitLit}). Stress-inducing localised pressure variations in the fluid cause mycelial fragmentation if the tensile strength of the hyphae is exceeded, while the peripheral regions of pellets can be sheared away if sufficient force is applied, dispersing the biomass and creating new centres of growth. Papagianni and colleagues demonstrated that increased agitation in batch cultures of \emph{A.~niger} can reduce hyphal vacuolation \cite{papagianni1999}. In batch cultures, both clump perimeter and filament length decreased with increasing agitation, while the percentage of vacuolated hyphal volume, as well as the mean diameter of vacuoles, were higher at lower agitation speeds. This suggested that higher levels of hyphal fragmentation occurred at higher agitation levels, which caused a reduction in vacuolated hyphae, indicating that fragmentation can aid in regrowth and limiting of vacuolation. However, high levels of agitation can cause excessive levels of mycelial damage, reducing biomass yield and compromising productivity \cite{elenshasy2006}.

\begin{table}[tbhp]
	\centering
	\footnotesize
	\caption{Reports in the literature indicating a link between mechanical agitation and morphology}
	\label{tab:AgitLit}
	\begin{tabularx}{(\textwidth - 1cm)}{l R c}
		\toprule
		Organism & Report & Reference \\ \midrule
		\emph{A. niger} & Increased agitation speed resulted in decreasing mean perimeter length of mycelial clumps. \lq Relative mixing time' proposed as significant process-controlling factor & \cite{papagianni1998}\\
		\emph{A. niger} & Increased agitation inhibited pellet growth in shake-flask cultures. Small pellets were associated with maximum phytase production & \cite{papagianni2001}\\
		\emph{A. niger} & Increased agitation caused reduction of pellets to \lq micro-pellets' and filamentous growth, resulting in increased glucose oxidase yield & \cite{elenshasy2006}\\
		\emph{A. niger} & Linear relationship between energy dissipation and pellet count per unit volume & \cite{kelly2004}\\
		\emph{A. oryzae} & Direct relationship derived between projected area of mycelial clumps and EDCF, while $\hgu$ was relatively independent of agitation intensity & \cite{amanullah1999}\\
		\emph{A. oryzae} & Amyloglucosidase production found to be approximately proportional to the number of \lq active' tips in a culture & \cite{amanullah2002}\\
		\emph{A. terreus} & Steady-state mean pellet diameter correlated with power input per unit volume. Pellets also became more compact at higher agitation speeds & \cite{rodriguezporcel2005}\\
		\emph{C. militaris} & Increased agitation resulted in smaller, more spherical (quantified in terms of circularity) pellets & \cite{jppark2002}\\
		\emph{T. harzianum} & Linear relationship between power input (EDCF function) and mean clump diameter & \cite{rocha-valadez2005}\\
		\emph{T. harzianum} & Reduction in the diameter of  mycelial clumps for increasing agitation intensity in shake flasks & \cite{galindo2004}\\
		\bottomrule
	\end{tabularx}
\end{table}

A relationship between agitation intensity and the growth form of \emph{A. niger} was described by Papagianni and colleagues, who proposed the \lq relative mixing time' (derived from analysis of pH response to an alkali pulse in both tubular-loop and stirred-tank reactors) as a significant process-controlling factor \cite{papagianni1998}. As agitation intensity was increased, the perimeter length of mycelial clumps decreased, while citric acid production increased. The increased citric acid yield was attributed to morphological variation, but reduced dissolved oxygen concentration was noted at lower agitation speed, which may have influenced metabolite production. Increased agitation was also found by Park and colleagues to cause a reduction in the size of \emph{Cordyceps militaris} pellets \cite{jppark2002}. Pellets were found to be larger and \lq fluffier' at low agitation intensity (50~rpm); at high agitation intensity (300~rpm), the outer filamentous regions were sheared off, resulting in a decrease in pellet size and circularity, defined as the ratio of the minimum Feret's diameter\footnote{The perpendicular distance between parallel tangents touching opposite sides of the perimeter} to the maximum. It was suggested that the larger pellets formed at low agitation intensity were probably oxygen-limited at their cores, resulting in autolysis at an earlier stage of fermentation and subsequent formation of hyphal \lq fragments'. Maximum exo-biopolymer production (\gl{15}) was obtained at 150~rpm, which was attributed to the compact, spherical nature of the pellets obtained at this agitation intensity.

The influence of shearing forces on pellet morphology was further investigated by Rodr\'{i}guez-Porcel and colleagues in fermentations of \emph{A.~terreus} \cite{rodriguezporcel2005}. Following inoculation of fluidised-bed reactors (FBR's) and stirred-tank reactors (STR's) operated at 300~rpm with a culture of pellets approximately 1,200~$\mu$m in diameter, the pellets grew to a steady state diameter of approximately 2,300~$\mu$m. However, for more intensely agitated STR cultures (600 and 800~rpm), pellet size declined from inoculation to an average steady state diameter of approximately 900 and 700~$\mu$m respectively, which was attributed to \lq erosion' of the pellet surface. Steady-state values of mean pellet diameter in all reactors was found to correlate well with total power input ($P_g$; including that from expansion of sparged gas) per unit volume ($P_g / V$). For all operating regimens and reactors, the filament ratio ([area of peripheral \lq hairy surface']/[total projected area]) declined to 0.4 during the first 72~hours of fermentation; agitation that did not result in a reduction in pellet size still caused a decrease in the filament ratio. This was attributed to fluid eddies \lq folding' the peripheral hyphae onto the pellet, compacting the structure. Highly agitated (800~rpm) STR cultures were an exception, in which a lower stable value of 0.2 was obtained.

The duration of shearing forces was found to be a significant morphological influence by Wongwicharn and colleagues in chemostat cultures of \emph{A. niger} \cite{wongwicharn1999}. Values of mean hyphal length, mean tip number and $\hgu$ all rose up to a dilution rate ($D$) of \h{0.08} before falling. This was explained by the greater residence time at low $D$; the average length of time hyphal elements were exposed to hydro-mechanical damage was greater than at high $D$, so in general a shorter, more compact \lq organism' resulted. Pellet core diameter also increased with $D$, due to increased growth rate and decreased mean residence time (less shear damage) as $D$ rose, which emphasises that duration of exposure to shearing forces can have a discernible effect on growth form.

The morphological influence of sudden \lq step' changes in agitation, and associated implications for metabolite production, were investigated by Paul and colleagues \cite{paul1999}. In a fermentation inoculated with pellets, diameter increased slightly (up to approximately 1.5~mm) before decreasing following a step increase in agitation from 500 to 800~rpm. Circularity increased slowly with time, indicating increasingly irregular pellets, probably due to fragmentation. During most of this fermentation the citric acid titre was comparable to that in another fermentation inoculated with spores and cultivated as a dispersed phenotype; evidence that the pellets were too small for significant diffusional limitations to have occurred. The overall yield however was greater from the dispersed fermentation (0.74 compared to 0.58~g~citric acid/g~glucose). A second fermentation inoculated with pellets was initially maintained at 300~rpm, which resulted in a rapid increase in diameter (up to 3.1~mm) with circularity increasing slowly - a step increase in agitation up to 500~rpm had little effect. A subsequent step increase to 800~rpm resulted in a rapid decrease in pellet diameter, accompanied by an increase in circularity, carbon dioxide production rate, oxygen uptake rate and a slight increase in growth rate, suggesting increased agitation caused a reduction in substrate-limited biomass. This fermentation resulted in the lowest yield of citric acid on glucose (0.49), probably owing to diffusional limitations in large pellets, reflected in a lower rate of glucose uptake, decreased specific growth rate, decreased specific uptake of oxygen and decreased specific production of carbon dioxide.

The work of Paul and colleagues suggests that increased agitation can result in a greater degree of biomass dispersal, a reduction in the fraction of unproductive biomass and, consequently, an increase in metabolite production. This hypothesis was further supported by El-Enshasy and colleagues, who reported a reduction in pellet size and an increase in the fraction of \lq active' biomass for increased agitation levels \cite{elenshasy2006}. Agitation speeds of 200, 500 and 800~rpm resulted in mean pellet diameters of 1,500, 400 and \mic{24} respectively, although the latter were described as \lq \emph{micro-pellets embedded in a filamentous mesh}', which may have complicated their measurement. The yield of glucose oxidase increased with increasing power input and the shift from pelleted to filamentous growth. Staining with acridine orange indicated that active protein production was restricted to free filamentous elements and the outer, peripheral region of pellets. It was also shown that the active region decreased in size with cultivation time in tandem with a decline in the specific glucose oxidase production rate. A comparison of cultures grown at 200 and 800~rpm showed similar specific glucose oxidase activities and production rates, when calculated based on the active fraction of biomass.

However, in their study of \emph{A. oryzae}, Amanullah and colleagues found that although increased agitation resulted in a decrease in the projected area of mycelial clumps, no increase in metabolite production was observed \cite{amanullah1999}. Steady-state values of mean projected area of biomass (clumps and freely dispersed elements) correlated well with the Energy Circulation/Dissipation Function (EDCF\footnote{$P/(K D^3 t_c)$, where $P$ is the power input, $D$ the impeller diameter, $t_c$, the mean circulation time, and $k$ is a geometric constant for a given impeller.}). However, no significant difference in the hyphal growth unit ($\hgu$) was measured and agitation speed did not have an appreciable effect on production rates (of $\alpha$-amylase and amyloglucosidase), although fewer tips per element were measured at higher agitation. It was proposed that because $\hgu$ was relatively independent of agitation speed, and assuming that the tip extension rate (which is coupled to enzyme synthesis) was also constant (influenced by dilution rate rather than agitation), the total number of \lq active' tips in the culture was approximately independent of agitation speed. This suggested that, within certain limits, agitation could be modified to regulate mixing and morphology without compromising metabolite production. In a subsequent study, Amanullah and colleagues used calcofluor white staining to distinguish between active and non-active tips in \emph{A. oryzae} \cite{amanullah2002}. A dependence of amyloglucosidase production on agitation intensity was found, which was attributed to the presence of a larger number of active tips at higher agitation speeds; an approximate correlation ($R^2 = 0.65$) between specific amyloglucosidase production and \% active tips was illustrated.

Further physiological influences of mechanical agitation were reported by Rocha-Valadez and colleagues in cultures of \emph{Trichoderma harzianum} \cite{rocha-valadez2005}. While a linear relationship between power input (EDCF function) and mean clump diameter was discerned, production of 6-Pentyl-$\alpha$-Pyrone, which increased with ECDF before decreasing, was not directly linked to the observed morphological shift. It was proposed that, given the observation of a linear increase in specific CO\sb{2} production and a linear decrease in specific growth rate, a metabolic shift had occurred in response to increased hydrodynamic stress, reducing biomass synthesis and metabolite production.

In addition to influencing hyphal fragmentation and shearing of pellet peripheral regions, mechanical agitation can also affect the extent of cellular aggregation. Amanullah and colleagues found that a \lq step-down' in agitation speed resulted in substantial aggregation of \emph{A.~oryzae} mycelia \cite{amanullah2001}. An increase in mean projected area from $17,800 \pm 1,500$ to \mics{$25,000 \pm 2,300$} and from $12,000 \pm 1,300$ to \mics{$17,000 \pm 2,000$}, at biomass concentrations of 5.3 and \gl{11.2} respectively, was observed between 5 and 15~minutes after a lowering of the EDC from 800 to 50~kWm\sp{3}s\sp{-1}. It was estimated that 85 -- 115~minutes would have been required if mycelial growth alone was responsible for these size increases. However, a significant increase in mean projected area was not measured in the absence of dissolved oxygen at either biomass concentration, suggesting that aggregation requires aerobic metabolism. It was also shown that an increase in the mean total length of the freely dispersed biomass fraction was faster than that in the mean projected area of clumps, leading the authors to speculate that small hyphal elements had a greater tendency to aggregate than larger elements, possibly due to their larger contact areas. These findings have implications for processes conducted in large-scale, aerated fermenters, as mycelia may be repeatedly fragmented and aggregated as they circulate through a large volume.

Variations in agitation intensity can also significantly affect spore agglomeration and subsequent pellet formation. Kelly and colleagues reported that increasing energy dissipation caused an initial decrease in the diameter and growth rate of \emph{A. niger} pellets \cite{kelly2004}. However, after reaching a minimum, the pellet diameter and the growth rate both increased again at a higher specific energy dissipation, but the overall concentration of pellets decreased linearly. It was suggested, perhaps counter-intuitively, that greater aggregation of spores at the beginning of cultivation due to higher agitation rates was responsible, resulting in fewer pellets. Each pellet therefore had a greater amount of substrate available and, in addition, higher agitation rates improved mass transfer, which together could have explained the increased growth rate and pellet diameter at higher energy dissipations. This assumes that increased agitation results in increased particle-to-particle interaction and that spore-spore \lq bonding' forces are sufficiently strong to withstand increased external shearing forces at higher agitation levels.

However, in their study of shake-flask cultures of \emph{A. niger}, Papagianni and colleagues reported that increased agitation resulted in the inhibition of pellet formation \cite{papagianni2001}. A fermentation conducted at 150~rpm resulted in a mixture consisting of pellets (average diameter 0.5~mm) and a small number of mycelial clumps. At higher agitation (300~rpm), no pellets were detected, with the culture consisting entirely of \lq loose' clumps and \lq free' mycelial trees, suggesting that increased agitation inhibited cellular aggregation. The effect of medium viscosity (regulated with the addition of guar gum) was also investigated and, at 150~rpm, pellet formation tended to reduce with increasing gum concentration; no pellets appeared at a gum concentration of \gl{1}. The morphological shift was attributed to the lower dissolved oxygen concentration in the presence of gum \cite{papagianni2001}.

\subsection{Carbon and nitrogen sources}

Certain nutrients are required by most fungi to maintain growth, but the general composition of the media can be altered to affect metabolism. This is common practice in industrial fermentation processes, as certain media components are often required in order to induce production of the metabolite of interest. An organic source of carbon and a source of nitrogen are almost always required; carbon typically constitutes approximately 50\% of dry-cell weight, while nitrogen is required to synthesise amino acids and proteins \cite{carlile2001}. While fungi are typically capable of producing a large number of enzymes for the extra-cellular degradation of substrates, sugars such as glucose, maltose and starch are commonly used as carbon sources, although some fungi can metabolise organic acids and even hydrocarbons \cite{prenafeta-boldu2006}. There also exists evidence that variations in media composition, and the subsequent influences on metabolism, can result in changes in morphology (Table~\ref{tab:CNLit}).

\begin{table}[t]
	\centering
	\footnotesize
	\caption{Reports in the literature indicating a link between medium composition and morphology}
	\label{tab:CNLit}
	\begin{tabularx}{(\textwidth - 1cm)}{l R c}
		\toprule
		Organism & Report & Reference \\ \midrule
		\emph{Aspergilli} & Correlation between volume of apical compartment and specific growth rate, regulated by glucose concentration & \cite{muller2000}\\
		\emph{A.~niger} & Increasing glucose concentration resulted in decrease in pellet diameter and increase in hyphal apical volume & \cite{papagianni2004}\\
		\emph{A.~niger} & Hyphae more susceptible to fragmentation at low glucose levels - increase in vacuolation also noted & \cite{papagianni1999}\\
		\emph{A.~oryzae} & Pulsed addition of carbon during fed-batch fermentation reduced mean projected area of mycelial elements, reducing broth viscosity & \cite{bhargava2005}\\
		\emph{M. alpina} & Increasing carbon to nitrogen ratio ($C/N$) resulted in increase in total pellet area ($A_m$) and size of pellet annular region ($A_f$) for $C/N > 20$, but $A_f/A_m$ remained constant & \cite{eypark2001}\\
		\emph{M. alpina} & Enriching media by increasing both $C$ and $N$ (with $C/N = 20$) resulted in decrease in pellet size & \cite{koike2001}\\
		\bottomrule
	\end{tabularx}
\end{table}

Papagianni and Mattey found that increasing glucose concentration resulted in a decrease in the diameter of \emph{A.~niger} pellets and an increase in citric acid production in immobilised cultures \cite{papagianni2004}. At low glucose concentration (\gl{50}), \lq hairy' pellets predominated with long, thin, unbranched filaments at the surface, but at higher concentrations (100 and \gl{150}), pellets were compact with short, thick filaments with swollen tips (based on SEM images). The rates of citric acid production increased with glucose level and were always higher with immobilized mycelium. The increased yield was attributed to the greater mean diffusion path in pellets compared to the immobilized system; immobilised biomass was considered to be maximally productive, while pellets were substrate-limited below the surface. However, the study suffered from a lack of quantitative methods for the microscopic analysis of filaments, although the routine imaging of immobilised mycelia may be challenging.

The manner in which substrate is added to a fermentation has also been determined to be an important influence on growth. Bhargava and colleagues have shown that the pulsed addition of substrate during fed-batch fermentations of \emph{A.~oryzae} can reduce the mean projected area of mycelial elements, reducing broth viscosity and increasing productivity \cite{bhargava2005}. Viscosity was found to be inversely proportional to cycle time ($t_c$; the length of time between pulsed additions of substrate) with cultures produced with a high value of $t_c$ exhibiting viscosities approximately 5 times lower than a continuously fed batch. The reduction in viscosity was likely a result of changes in morphology, with the average size of fungal elements (clumps or free mycelia; measured by projected area) decreasing approximately linearly with $\log t_c$. However, this structural variation did not result in a significant impact on glucoamylase production (for $t_c \leq 300$~s), indicating that culture viscosity could be reduced (reducing requisite power input) without compromising metabolite production. However, for longer cycle times ($t_c \geq 900$~s), a marked reduction in glucoamylase levels was observed (approximately 50\%), which was likely related to the increased occurrence of sporulation noted by the authors, caused by low substrate levels. Low levels of carbon substrate have also been demonstrated to result in increased hyphal vacuolation and fragmentation in fed-batch cultures of \emph{A. niger} \cite{papagianni1999}. Reducing the glucose concentration from 70 to \gl{17} resulted in an increase in the percentage vacuolated volume of hyphae as well as a reduction in the mean length of hyphal filaments, suggesting an increase in fragmentation at lower glucose concentrations.

M\"{u}ller and colleagues also quantified a micro-morphological and physiological influence of substrate concentration in cultures of \emph{A.~oryzae} and \emph{A.~niger} \cite{muller2000}. The average number of nuclei in the apical compartment was found to increase for increasing specific growth rate, but little or no change was observed in sub-apical compartments. The average length and diameter of an apical compartment also increased for increasing specific growth rate (the length of sub-apical compartments was only slightly affected); this increase in volume may have been required to accommodate an increase in cytoplasmic mass, so as to maintain a constant cytoplasm to nucleus ratio. An increase in apical volume in response to increased glucose concentration was also noted by Papagianni and Mattey in cultivations of \emph{A.~niger} \cite{papagianni2004}.

While it has been suggested that a balanced medium will contain ten times as much carbon as nitrogen \cite{carlile2001}, varying this ratio can have a significant influence on morphology. Park and colleagues found that increasing the carbon to nitrogen ratio ($C/N$) in cultivations of \emph{Mortierella alpina} resulted in both an increase in total pellet area, $A_m$, and an increase in the size of the pellet annular region, $A_f$ (for $C/N > 20$), but the filamentous fraction ($A_f/A_m$) remained constant \cite{eypark2001}. The pellet core size was 0.45~mm\sp{2} on average, and did not change for $C/N < 20$, with the filamentous mycelial area displaying a similar trend (1.1~mm\sp{2} on average). The ratio of the filamentous mycelial area to the whole mycelial area was 0.82 and was independent of the consumed $C/N$ ratio. When the medium was enriched 3-fold, the filamentous mycelial area and pellet core size increased 8.6- and 4.7-fold compared to the control. The ratio of the filamentous mycelial area to the whole mycelial area was 0.76, which was independent of the consumed $C/N$ ratio despite the use of enriched media.

However, Koike and colleagues found that enriching medium at a fixed $C/N$ ratio resulted in a decrease in the whole pellet size and the width of the pellet annular region in \emph{M. alpina}, but variations in morphology similar to those described by Park and colleagues were reported for variations in the $C/N$ ratio \cite{koike2001}. The whole pellet size ($A_m$) did not change for $C/N < 20$, but then increased gradually with an increase in the consumed $C/N$ ratio. The width of the annular region remained approximately constant for $C/N < 20$ and then increased linearly with increasing $C/N$ ratio.

\subsection{Culture pH}

The medium pH is an important environmental factor that can significantly influence the outcome of a fermentation. While filamentous fungi can grow over a wide range of pH (typically 4 -- 9), maximal growth is often found near neutral pH \cite{carlile2001} and minimising pH drift is often a target in bioprocess design. However pH has also been shown to impact morphology and, in particular, pellet formation (Table~\ref{tab:pHLit}).

\begin{table}[htbp]
	\centering
	\footnotesize
	\caption{Reports in the literature indicating a link between pH and morphology}
	\label{tab:pHLit}
	\begin{tabularx}{(\textwidth - 1cm)}{l R c}
		\toprule
		Organism & Report & Reference \\ \midrule
		\emph{A.~nidulans} & Reduction in electrophoretic mobility of conidiospores for increasing pH, which was subsequently shown to increase pellet formation & \cite{dynesen2003}\\
		\emph{A.~oryzae} & At pH~3.0 -- 3.5 freely dispersed hyphal elements predominated, with pellets becoming more common as acidity was reduced & \cite{carlsen1996a}\\
		\emph{Phellinus} & Macro-morphological effect of pH variation on different species, but not expressed quantitatively & \cite{hwang2004}\\
		\emph{R. nigricans} & Growth suppressed below pH~2. Clumpy growth noted at pH~3 or above 7, otherwise pelleted growth prevailed & \cite{znidarsic2000}\\
		\bottomrule
	\end{tabularx}
\end{table}

In batch cultivations of \emph{A.~oryzae}, Carlsen and colleagues reported that at pH~3.0 -- 3.5 freely dispersed hyphal elements predominated, with pellets becoming more common as acidity was reduced \cite{carlsen1996a}. At very low pH ($\leq 2.5$) the mycelium was vacuolated with swollen cell walls, resulting in poor growth, but at pH~3.0 to 3.5, freely dispersed hyphal elements resulted. As pH was increased further, pellets became more prevalent, with no free elements discernible at high pH ($\geq 6$) and pellet radius increasing with medium alkalinity. At relatively high pH ($> 4$), agglomerates containing 10 to 100 spores were observed approximately 10~hours post-inoculation, whereas at low pH (2.5 to 3.5) only freely dispersed spores were found. When cultivations were inoculated with freely dispersed hyphal elements no pellets were witnessed and it was therefore concluded that pellet formation in \emph{A. oryzae} resulted from coagulation of spores, which was dependent on medium pH. While no micro-morphological influence of pH was found (based on measures of total hyphal length and number of tips per element), the effect of pH on the growth kinetics was studied and a broad optimum between pH~3 and 7 was found, while the specific $\alpha$-amylase production rate had an optimum in the range pH~5 to 7.

Dynesen and Nielsen demonstrated a reduction in the electrophoretic mobility of \emph{A.~nidulans} conidiospores for increasing pH, which was subsequently shown to increase pellet formation, although the influence of hydrophobicity was also reported as significant \cite{dynesen2003}. Contact angles\footnote{The contact angle is the angle at which a liquid/vapour interface meets a solid surface.} (determined using water droplets on layers of conidiospores) for strains lacking hydrophobins were lower than those of the control strain and for all strains, electrophoretic mobility decreased from positive to negative with increasing pH. The percentage of free mycelia in shake flasks 19~hours post-inoculation increased with decreasing pH and for all pH values studied (except pH~5.8), the percentage of free mycelia increased with decreasing hydrophobicity. Disruption of genes encoding hydrophobins did not affect the surface charge of the conidiospores, so the reduced pellet formation by the mutant strains could not be attributed to differences in electrostatic interactions and it was therefore suggested that a high surface hydrophobicity of \emph{A. nidulans} conidia favoured pellet formation. However, since pellet formation by the mutant strain exhibiting the lowest hydrophobicity was observed at low pH values, the authors concluded that agglomeration of \emph{A. nidulans} conidia could not be attributed to hydrophobic and electrostatic interactions alone; possible chemical interactions such as polysaccharide bridging, may also have been at play.

Hwang and colleagues displayed the macro-morphological effect of pH level on different \emph{Phellinus} species, although the variation was not expressed quantitatively \cite{hwang2004}. \emph{P. baumii} produced high amounts of free mycelium as the culture pH shifted to neutral and alkaline ranges. In contrast, \emph{P. gilvus} displayed pelleted growth at low pH (4 and 5) whereas free mycelia were predominant at pH~9. \emph{P. linteus} showed spherical pellet growth with lesser amounts of free mycelia irrespective of culture pH.

\subsection{Supplementation with metal ions}

Certain essential metals are required for fungal growth and their absence from nutrient media can result in developmental abnormalities, while excess levels of certain ions can induce morphological variation (Table~\ref{tab:IonLit}). For example, omission of Mn\sp{2+} ions were found to have a significant impact on the morphology of \emph{A.~niger} by Kisser and colleagues, although the assessment of variation in form was limited to qualitative descriptions \cite{kisser1980}. In the presence of manganese ($5 \times 10^{-5}$~M), growth was initiated by the emergence from spores of germ tubes after 18 -- 20~hours. The germ tubes were thin, had very few branches (described as \lq filamentous' growth) and citric acid production was poor (44~mM after 150~h). However, in the absence of manganese, considerable swelling of spores occurred prior to germ tube formation and hyphae were characterised as \lq squat, bulbous cells', which were preferable for a high citric acid yield (286~mM). An intermediate level of manganese ($4 \times 10^{-7}$~M~L\sp{-1}) resulted in both the \lq filamentous' and \lq bulbous' phenotypes. The addition of iron ($> 5 \times 10^{-4}$~M) also resulted in filamentous growth in the absence of manganese, while the addition of copper ($10^{-4}$ -- $10^{-3}$~M) reduced the influence of manganese.

\begin{table}[t]
	\centering
	\footnotesize
	\caption{Reports in the literature indicating a link between metal ion supplementation and morphology.}
	\label{tab:IonLit}
	\begin{tabularx}{(\textwidth - 1cm)}{l R c}
		\toprule
		Organism & Report & Reference \\ \midrule
		\emph{A. niger} & Omission of Mn\sp{2+} ions resulted in \lq \emph{squat, bulbous hyphae}' &  \cite{kisser1980}\\
		\emph{A. niger} & Addition of Fe\sp{2+} and Zn\sp{2+} ions resulted in diffuse pellets & \cite{couri2003}\\
		\emph{R. nigricans} & Increase in mean pellet diameter observed in presence of Ca\sp{2+} ions & \cite{znidarsic2000}\\
		\emph{S. hygroscopicus} & Cultures supplemented with Ca\sp{2+} ions were hydrophobic, resulting in cellular aggregation. Addition of Mg\sp{2+} ions resulted in hydrophilic cells and freely dispersed filaments & \cite{dobson2008a}\\
		\bottomrule
	\end{tabularx}
\end{table}

The addition of metal ions has also been used to influence the hydrophobicity of cells, which subsequently affected cellular aggregation. Dobson and O'Shea found that the hydrophobicity of \emph{Streptomyces hygroscopicus} cells could be influenced with the addition of calcium or magnesium ions \cite{dobson2008a}. Cultures supplemented with Ca\sp{2+} ions were found to be hydrophobic, which resulted in cellular aggregation, while the addition of Mg\sp{2+} ions resulted in hydrophilic cells with the organism growing as freely dispersed filaments, with shorter, more branched hyphae visible, although these dispersed elements were not morphologically analysed. Increasing the concentration of Ca\sp{2+} from 0.01 to \gl{0.5} resulted in a 40\% increase in pellet diameter and a decrease of approximately 70\% in the number of pellets per ml, which reflected the extent of aggregation. This increase in pellet size coincided with a decrease in geldanamycin yield of 85\%. Evaluation of the cell-surface hydrophobicity (CSH), by measuring of contact angles, determined that control cultures were hydrophilic (contact angle of $37.65 \pm 4.1^{\circ}$), whereas pelleted cultures containing Ca\sp{2+} ions were strongly hydrophobic ($76.64 \pm 5.5^{\circ}$). In contrast, Mg\sp{2+}-supplemented cultures growing as dispersed filaments were classed as hydrophilic ($22.89 \pm 4.9^{\circ}$). A relationship between pellet size and CSH was demonstrated, with increasing hydrophobicity instigating aggregation and pellet formation.

Metal ion supplementation was also explored by Couri and colleagues in cultures of \emph{A.~niger}, who found that the addition of Fe\sp{2+} and Zn\sp{2+} ions resulted in diffuse pellets compared to the control \cite{couri2003}. Pellet diameter ($D$), perimeter and pellet core diameter ($d$) were measured manually and other measures, such as pellet area ($A_p$) and pellet core area ($A_c$), were derived. The addition of Fe\sp{2+} and/or Zn\sp{2+} resulted in smaller pellets with smaller cores, while the ratio of $D/d$ was higher and $A_c/A_p$ lower in pellets incubated in the presence of both ions, compared to those fermentations supplemented with just one ion. Polygalacturonase production was higher in the presence of both ions compared to the control, but specific activity was higher in control cultures, even though morphologies were similar. However, control pellets were larger and probably autolysed at the core (based on measures of weight and volume), which possibly biased measures of specific activity.

\subsection{Supplementation with surface active agents and polymers}

Various surface active agents (surfactants) have also been investigated in attempts to disrupt cellular aggregation in submerged processes. \v{Z}nidar\v{s}i\v{c} and colleagues found that the supplementation of \emph{Rhizopus nigricans} cultures with Tween-80 caused a slight increase in pellet diameter, but higher concentrations (1 -- \gl{2}) also produced some clumped growth \cite{znidarsic2000}. The increase in pellet size, which was accompanied by an increase in biomass, was attributed to increased cell membrane permeability, facilitating more efficient mass transfer of nutrients. Other factors were also investigated, such as culture pH and the influence of Ca\sp{2+} ions, but the morphological assessment was rather qualitative in nature, with pellets classified as either \lq smooth' or \lq fluffy'. Conversely, Domingues and colleagues reported that Tween-80 inhibited pellet formation in \emph{T.~reesei} \cite{domingues2000}. Inoculation with \sm{$10^5$} resulted in pellet formation in a control media, but filamentous growth resulted in the presence of Tween-80, although only qualitative morphological descriptions were produced. A more concentrated inoculum (\sm{$10^7$}) resulted in filamentous growth in both media. Supplementation with Tween-80 also resulted in elevated yields of both protein concentration and cellulase production at both inoculum concentrations, with increased permeability of the cell membrane again suggested as the cause, allowing more rapid secretion of enzymes which in turn leads to higher enzyme synthesis.

\begin{table}[t]
	\centering
	\footnotesize
	\caption{Reports in the literature indicating a link between surfactant and/or polymer supplementation and morphology}
	\label{tab:SurfactantPolymerLit}
	\begin{tabularx}{(\textwidth - 1cm)}{l R c}
		\toprule
		Organism & Report & Reference \\ \midrule
		\emph{A.~niger} & Addition of kaolin resulted in small, compact pellets & \cite{ali2006}\\
		\emph{A. niger} & Carboxymethylcellulose shifted morphology from pelleted to filamentous & \cite{ahamed2005}\\
		\emph{R. nigricans} & Supplementation with Tween-80 caused slight increase in pellet diameter & \cite{znidarsic2000}\\
		\emph{S. hygroscopicus} & Supplementation with Tween-80 had minimal influence on morphology, but Triton X-100 resulted in significant increase in pellet size & \cite{dobson2008b}\\
		\emph{S.~hygroscopicus} & Concentrations of carboxymethylcellulose up to 3.0\% resulted in decrease in pellet size. Reduction in \lq wall growth' also observed & \cite{ilic2008}\\
		\emph{T. reesei} & Tween-80 inhibited pellet formation & \cite{domingues2000}\\
		\emph{T. harzianum} & Tween-40 resulted in reduction in pellet size, with more dispersed growth & \cite{lucatero2004}\\
		\bottomrule
	\end{tabularx}
\end{table}

Supplementation of \emph{S.~hygroscopicus} cultures with Tween-80 was reported by Dobson and colleagues to have a minimal impact on morphology, but the inclusion of 0.01\% (v/v) Triton X-100 resulted in a considerable increase in pellet size \cite{dobson2008b}. Further increases in the concentration of Triton X-100 (up to 1.0\% v/v) resulted in a subsequent decrease in pellet size (to control levels). While the addition of Tween-80 did not have a significant influence on pellet size, an increase in concentration did result in an increase in pellet count per ml, which was accompanied by an increase in geldanamycin production. However, control cultures produced a greater or equal yield of geldanamycin compared to all surfactant-supplemented cultures. Silicone antifoam was also found to be influential, with an increase in the concentration present in the medium causing an increase in the dispersion of pellets; pellet sizes decreased by more than 50\% in those cultures supplemented with 5\% (v/v) silicone antifoam. Antibiotic synthesis appeared to be repressed by the formation of large pellets, implying that cultures with smaller pellet sizes are optimal for geldanamycin production.

Supplementation of \emph{T.~harzianum} cultures with Tween-40 also resulted in a reduction in pellet size, together with more dispersed growth, and a linear relationship between Tween-40 concentration and biomass yield was also reported \cite{lucatero2004}. In the absence of Tween-40, large, \lq star-like' pellets were formed, while supplementation with 0.2~ml~L\sp{-1} Tween-40 resulted in the formation of spherical pellets, typically exhibiting a smooth, compact structure and a narrow filamentous outer region, reflected in the lower roundness value. With respect to the control, a narrower particle size distribution resulted in the presence of Tween-40, indicating greater pellet homogeneity. A rapid transition from pelleted to dispersed/clumped morphologies at Tween-40 concentrations above 0.4~ml~L\sp{-1} was observed, although a small number of pellets were still present at this concentration. It was suggested that high Tween-40 concentrations may have resulted in the incorporation of Tween molecules into the cell wall, influencing spore aggregation.

Other studies have explored the influence of polymers on culture conditions. Ili\'{c} and colleagues described the effect of carboxymethylcellulose (CMC) supplementation on cultures of \emph{S.~hygroscopicus}, with concentrations of up to 3.0\% resulting in a decrease in pellet size, a significant increase in biomass levels and a reduction in wall growth \cite{ilic2008}. In the absence of CMC, large pellets ($\sim 4.1$~mm in diameter) developed and \lq \emph{extra heavy}' wall growth was observed. With the addition of CMC, pellet diameter decreased to a minimum of approximately 0.7~mm at a concentration of 3.0\% (w/v) and wall growth was reduced to low levels. An increase in metabolite production in the presence of CMC was also noted, with the maximum concentration of hexaene H-85 obtained at 3.0\% CMC (146.7~mg~dm\sp{-3} versus 94.58~mg~dm\sp{-3} in control culture) and the maximum concentration of azalomycine was obtained at 1.0\% CMC (188.6~mg~dm\sp{-3} versus 115.7~mg~dm\sp{-3} in control culture). It was therefore concluded that smaller pellets favoured elevated antibiotic production. However, pellet diameters were determined in a crude manner, with measurements obtained from microscope images using a ruler; only ten pellets were analysed for each sample. CMC supplementation was also reported to induce filamentous growth in \emph{A. niger}, but the morphological variation was not assessed quantitatively \cite{ahamed2005}.

O'Cleirigh and colleagues regulated the apparent viscosity of \emph{S.~hygroscopicus} broths with the addition of xanthan gum, increased concentrations of which (up to 3~g/L) resulted in a decrease in pellet size \cite{ocleirigh2005}. The pellet count increased to a maximum (by a factor of 4 relative to the control) at a concentration of \gl{3} xanthan gum and then decreased in excess of that point, whereas the mean pellet volume decreased (by a factor of 3.5) with respect to increasing xanthan gum concentration. An increase in biomass production of up to 2.5-fold was also recorded. This data suggested that, by inhibiting particle aggregation, the development of individual spores into smaller pellets was achieved. It was also found that the addition of xanthan gum resulted in an increase in gas-liquid mass transfer ($K_\mathsf{L} a$), increasing the oxygen transfer to pellets. Increased broth viscosity was also determined to inhibit pellet growth in \emph{A.~niger}, but reduced dissolved oxygen concentration was assumed to be responsible \cite{papagianni2001}.

The addition of kaolin\footnote{A common mineral (also known as kaolinite) widely referred to as \lq China clay'.} to cultures of \emph{A.~niger} was reported by Ali to promote the formation of small, compact pellets \cite{ali2006}. When a 15~ppm aqueous suspension of kaolin was added to the fermentation, citric acid yield peaked at \gl{74.62} 96~hours post-inoculation, compared to a peak of \gl{40.94} 168~hours post-inoculation in the control culture, with morphology shifting from \lq mixed mycelia' (1.10 -� 1.75~mm in size) to \lq small pellets' (1.12~mm). Process performance was further enhanced by delaying the addition of kaolin (15~ppm) until 24~hours post-inoculation, which resulted in a citric acid yield of  \gl{96.88}, which resulted in a \lq mixed pellet' (1.08 -� 1.28~mm) growth form. However, the methodology employed in determining \lq mycelial size' was not specified.

\subsection{Significance of branching}

In investigating the role of different environmental variables in microbial development, many researchers have demonstrated reproducible relationships between macro-morphology and product yield for a particular process (Table~\ref{tab:MorphProdLit}). Often, however, micro-morphological variation has been overlooked. Fundamental to furthering the understanding of morphological influence on product yield is the eliciting of a link between hyphal branching and metabolite production, as there is considerable evidence in the literature that protein secretion occurs almost exclusively at the hyphal apex (Table~\ref{tab:BranchProdLit}).

\begin{table}[t]
	\centering
	\footnotesize
	\caption{Recent reports in the literature indicating a relationship between macro-morphology and productivity}
	\label{tab:MorphProdLit}
	\begin{tabularx}{(\textwidth - 1cm)}{l R c}
		\toprule
		Organism & Report & Reference \\ \midrule
		\emph{A. niger} & Small pellets optimal for minimising proteases and maximising green fluorescent protein yield & \cite{xu2000}\\
		\emph{A. niger} & Small pellets preferable for polygalacturonase synthesis & \cite{couri2003}\\
		\emph{A. niger} & Reduced pellet size provided increase in glucose oxidase production & \cite{elenshasy2006}\\
		\emph{A. niger} & Large pellets associated with maximal glucosamine production & \cite{papagianni2006a}\\
		\emph{A. oryzae} & Filamentous growth optimal for $\alpha$-amylase production & \cite{carlsen1996a}\\
		\emph{C. militaris} & Compact pellets favourable for exo-biopolymer production & \cite{jppark2002}\\
		\emph{P. chrysogenum} & Filamentous growth favourable for penicillin production & \cite{elsabbagh2006}\\
		\emph{R. chinensis} & Mycelium-bound lipase production increased from 101.2 to 691~U~g\sp{-1} with a change in morphology from dispersed mycelia to large agglomerates & \cite{teng2009}\\
		\emph{S. hygroscopicus} & Dispersed growth synthesised geldanamycin at an optimal rate & \cite{dobson2008a}\\
		\bottomrule
	\end{tabularx}
\end{table}

\begin{table}[t]
	\centering
	\footnotesize
	\caption{Reports in the literature indicating a link between branch formation and productivity}
	\label{tab:BranchProdLit}
	\begin{tabularx}{(\textwidth - 1cm)}{l R c}
		\toprule
		Organism & Report & Reference \\ \midrule
		\emph{A. niger} & Localisation of glucoamylase secretion at hyphal tips & \cite{wosten1991}\\
		\emph{A. niger} & High citric acid productivity characterised by swollen hyphal tips & \cite{papagianni2004}\\
		\emph{A. niger} (B1-D) & Correlation between \lq{}percentage active length' of hyphae and total soluble protein concentration & \cite{wongwicharn1999}\\
		\emph{A. oryzae} & Correlation between amyloglucosidase production and number of active tips & \cite{amanullah2002}\\
		\emph{A. oryzae} & Densely branched strain favourable for $\alpha$-amylase production & \cite{spohr1997}\\
		\emph{A. oryzae} & Localisation of $\alpha$-amylase in apical cell walls & \cite{muller2002}\\
		\emph{A. oryzae} & Strains with lower $\hgu$ values produced higher yields of $\alpha$-amylase, glucoamylase and proteases & \cite{tebiesebeke2005}\\
		\emph{A. oryzae} & Lipase production coincided with swelling of hyphal tips & \cite{haack2006}\\
		\emph{P. cinnabarinus} & Correlation between branching and phenol-oxidase secretion & \cite{jones1997}\\
		\bottomrule
	\end{tabularx}
\end{table}

W\"{o}sten and colleagues developed a technique for the localisation of glucoamylase around the hyphal tips of \emph{A.~niger} \cite{wosten1991}. A polycarbonate membrane was centrally placed on the surface of solidified medium and topped with an agarose layer (approx. 0.3~mm thick), which was inoculated with a small piece of mycelium. The surface was then topped with a second polycarbonate membrane, causing colonies to grow approximately two-dimensionally. When 4-day-old colonies were transferred from xylose to starch and incubated overnight, starch-degrading activity (detected with Lugol's iodine) was localized at the colony periphery; prolonging growth for another 8~hours resulted in virtually all starch under the colony being degraded. It was also found (using immunodetection) that enzymes were mainly secreted at the periphery of the colony; glucoamylase was observed around the tips of leading hyphae, but little was seen surrounding subapical regions. When colonies were transferred to a cold environment (\celc{4}) for 10~minutes before being re-transferred to the original plate at \celc{25}, labelling with N-acetylglucosamine showed that all hyphae had stopped growing and immunosignals of glucoamylase could no longer be detected. No interference with growth and secretion occurred when colonies were transferred to a fresh agar medium at \celc{25} and after 10~minutes re-transferred to the original plate. It was therefore concluded that glucoamylase was secreted exclusively by actively growing tips of \emph{A. niger}.

However, M\"{u}ller and colleagues contended that during growth on solid medium, secreted proteins will always be found surrounding the hyphae and, therefore, studies localising metabolite excretion in mycelia grown on agar are inconclusive \cite{muller2002}. Their work focussed on submerged culturing of \emph{A. oryzae} and FITC labelling of antibodies suggested that $\alpha$-amylase resided in the cell walls of hyphae. Fluorescence was greatest in new tips or extending hyphae, while older hyphae (more than \mic{100} from the tip) did not fluoresce as strongly, supporting the hypothesis that $\alpha$-amylase is secreted from the hyphal apex. However, a comparison of three different strains of \emph{A. oryzae} showed similar levels of amylase productivity, despite differing branching patterns; the $\hgu$ of a mutant strain was 52\% lower than a wild-type. However, it was also determined that the estimated maximum tip extension rate and average tip extension rate were reduced by 20 -- 80\% in the mutant strains, which may explain the lack of an increase in amylase yield for increased branching. A similar conclusion was reached by Johansen and colleagues in their study of \emph{A. awamori} \cite{johansen1998}. A higher inoculum concentration resulted in lower values of $\hgu$ and lower tip extension rate; if enzyme secretion is coupled to growth (as reported by Carlsen and colleagues in the case of $\alpha$-amylase production by \emph{A. oryzae} \cite{carlsen1996b}), this would result in less metabolite production per tip, offsetting the increase in the number of tips per mycelium.

Further evidence of metabolite secretion at hyphal tips was reported by Haack and colleagues, who observed changes in apical volume in \emph{A.~oryzae} in response to elevated production of lipase \cite{haack2006}. During the fed-batch phase of fermentation, during which high lipase production was evident, the ratio between the diameter of the hyphal tip and the diameter of the hypha (measured \mic{20} from the apex) increased from less than 1.0 during the exponential phase to 2.5 at the end of the fed-batch phase. A cessation of cell growth and lipase production were coincident with a return to a normal diameter of hyphal tip. Changes in apical volume have also been described in \emph{A.~niger} in response to elevated production of citric acid \cite{papagianni2004}. 

More significantly, other studies have documented associations between the extent of mycelial branching and productivity. Jones and Lonergan proposed a link between the branching complexity (quantified as the fractal dimension) of \emph{Pycnoporus cinnabarinus} and phenol oxidase expression \cite{jones1997}. Solid media was supplemented with different concentrations of Remazol Brilliant Blue R (RBBR) to induce variation in branching patterns and it was reported that higher $D$ values correlated with increasing tip and sub-apical branching. A trend was observed in $D$ values for differing RBBR concentrations, although the associated errors in the mean $D$ values were relatively large (and often overlapped), possibly due to the small number of colonies analysed at each RBBR concentration ($\geq 10$). However, variation in oxidase activity in submerged culture, in response to differing RBBR concentrations, followed similar trends to fractal dimension in solid culturing, demonstrating a positive correlation between branching complexity and oxidase enzyme expression.

A proxy indicator of branching complexity was also employed by Papagianni and colleagues, who demonstrated a dependency between citric acid titre and the length of filaments at the periphery of \emph{A.~niger} mycelial clumps, independent of the reactor used for cultivation (stirred-tank or tubular loop reactor) \cite{papagianni1994}. Decreasing the circulation time in either a tubular loop reactor (TLR) or a stirred tank reactor (STR) resulted in an increase in citric acid titre, while clump perimeter ($P1$) and the length of filaments arising from the clump core ($l$) both decreased. A correlation between hyphal diameter ($d$) and citric acid titre was also evident in the TLR, although no clear relationship was identified in the STR. Since $l$ had a direct influence on $P1$, it was therefore concluded that a single parameter, $l$, could be directly coupled to citric acid production.

A more direct link between metabolite production and hyphal tips was demonstrated by Wongwicharn and colleagues in deriving a correlation between the \lq active length' of hyphae (determined by calcofluor white staining \cite{gull1974}) and total soluble protein concentration in chemostat cultures of \emph{A.~niger} \cite{wongwicharn1999}. A close relationship between the mean number of tips versus total extracellular protein was observed, while a linear dependency of total extracellular protein on mean total concentration of tips (biomass concentration $\times$ mean number of tips per organism) was found; a link between \% active length and protein secretion was also derived. A similar result was derived by Amanullah and colleagues between the number of active tips of \emph{A.~oryzae} and amyloglucosidase production \cite{amanullah2002}. 

Other studies have suggested that morphological mutants are preferable for high levels of protein secretion. For example, Spohr and colleagues observed increased productivity in a densely-branched mutant strain of \emph{A.~oryzae} compared to a wild-type \cite{spohr1997}. Comparison of two recombinant strains showed that the morphological mutant exhibited substantially higher $\alpha$-amylase production, possibly indicating that changes in branching behaviour influenced enzyme production; a dense mycelium with many tips may be able to secrete protein at a higher rate than a less branched mycelium. It is possible that physiological changes may have been responsible for this increase in production, although it was concluded that based on the combination of physiological and morphological characterization presented, there did not seem to be a significant difference in the physiology of the two strains. Increased productivity in densely-branched mutant strains of \emph{A.~oryzae} (compared to a wild-type) was also recognised by te Biesebeke and colleagues \cite{tebiesebeke2005}. The different strains were cultivated for up to 24~hours in a 1 -- 2~mm layer of wheat-based solid medium (WSM) and the mutant strains exhibited an average $\hgu$ 50 -- 74\% that of the wild type. When subsequently grown on wheat kernels (WK), the $\alpha$-amylase activities measured in the media extracts were at least 50\% higher in the mutant strains compared to that of a wild-type, glucoamylase activities more than 100\% higher and protease activities more than 90\% higher. Increased biomass did not seem likely to be responsible for the increase in protein production, as biomass yields on WSM were similar for all strains. However, the manner in which the morphology was quantified on WSM was not made clear, the authors stating that analysis was conducted using ImageJ \cite{imagej} \lq \emph{according to the manufacturer's protocol}'.

However, there are other reports of a disconnect between branching complexity and metabolite production in certain processes. For example, Jayus and colleagues contended that hyphal branching frequencies had no discernible effect on productivity in fermentations of \emph{Acremonium} sp. IMI 383068 \cite{jayus2005}. $\hgu$ fell from \mic{$93 \pm 13$} to \mic{$64 \pm 16$} as the agitation speed was increased from 100 to 200~rpm respectively, but remained approximately constant (\mic{$55 \pm 14$}) as agitation was increased up to 600~rpm. For the same increases in agitation speed, ($1 \rightarrow 3$)-$\beta$-glucanase yields fell substantially. However, the manner in which the morphology was quantified was not presented. Furthermore, the authors described difficulty in differentiating between mycelial branches representing \lq true vegetative hyphae' and those that would eventually form conidiophores, which represents a potential source of error in the estimations of $\hgu$.

Other authors have hypothesised that, in certain filamentous microbes, metabolite excretion does not occur at hyphal tips. Martin and Bushell proposed that antibiotic secretion from \emph{Saccharopolyspora erythraea} occurred at sites a fixed distance \emph{behind} the advancing hyphal tip, rather than from the tip itself \cite{martin1996}. By comparing the specific erythromycin yield from two different mycelial populations with different size distributions, it was concluded that a minimum particle diameter of \mic{88} (determined by circumscribing circles around images of mycelia) was required for antibiotic production - \lq \emph{particles below the critical size would grow but not produce antibiotic until they had increased in size sufficiently to include the hypothetical antibiotic secretion site}'. It was also found that particle size distributions were significantly influenced by agitation speed, with the mean diameter increasing from 70 to \mic{124} when agitation was reduced from 1,500 to 750~rpm, causing an increase in specific antibiotic production rates from 0.867 to \h{0.913~mg~g\sp{-1}}.