\section{Industrial fermentation of fungi}

Filamentous microbes, and fungi in particular, are perhaps traditionally associated with destructive and parasitic behaviour, incurring immense economic losses in the process. Their invasive growth strategies enable colonisation of a wide variety of plants, trees, insects, fish, animals and even humans. Some of the most devastating crop failures in recorded history have been attributed to fungal disease, such as the Irish potato famine of 1845 (\emph{Phytophthora infestans}) and the Bengal rice famine of 1943 (\emph{Cochliobolus miyabeanus}).

However, the saprophytic mode of nutrition of many filamentous microbes has been exploited to produce a wide range of compounds of economic significance, such as enzymes, antibiotics, plant growth regulators and vitamins (Table~\ref{tab:MetaInd}). This list is by no means exhaustive and in fact the range of products obtainable from filamentous fungi is frequently expanded upon. For example, investigations into the use of fungi for the production of diesel-like compounds (\lq myco-diesel') are well underway \cite{adamczak2009}, while the utilisation of novel cultivation formats can result in increased productivity \cite{papagianni2004} or even the discovery of new metabolites of potential utility \cite{bigelis2006}. The future economic potential of industrial biotechnology has been outlined in several reports, with the British Department of Business Enterprise and Regulatory Reform (BERR) recently estimating that the global market for bio-based products will grow to US\$250~billion by 2020 \cite{berr2009}.

\begin{table}[htbp]
	\centering
	\footnotesize
	\caption{A selection of metabolites produced in industry by fungi, their uses and some of the organisms used in their production \cite{archer2001,carlile2001,papagiannireview}.}
	\label{tab:MetaInd}
	\begin{tabularx}{(\textwidth - 1cm)}{R R R}
		\toprule
		Metabolite & Uses & Producer organisms \\ \midrule
		Biomass & Food industry & \emph{A. bisporus}, \emph{F. venenatum}\\
		Amylases & Food industry, detergents & \emph{A.~niger}, \emph{A.~oryzae}, \emph{A.~awamori}, \emph{R.~oryzae}, \emph{T.~viride}\\
		Glucoamylases & Food industry & \emph{A.~niger}, \emph{A.~awamori}, \emph{A.~oryzae}, \emph{R.~oryzae} \\
		Cellulases & Food processing, pulp \& paper industry & \emph{A.~niger}, \emph{A.~terreus}, \emph{T.~reesei}, \emph{T.~viride} \\
		Lipases & Detergents & \emph{A.~niger}, \emph{A.~oryzae}, \emph{R.~arrhizus} \\
		Phytases & Animal feedstuffs & \emph{A.~niger} \\
		Ethanol & Chemical industry, biofuels & \emph{S.~cerevisiae} \\
		Organic acids (citric, gluconic, itaconic) & Food, soft drinks and pharmaceutical industries & \emph{A.~niger}, \emph{A. terreus}, \emph{T.~viride}\\
		Riboflavin & Food supplement, various clinical applications & \emph{A.~gossypii} \\
		Penicillins & Antibiotics & \emph{P.~chrysogenum} \\
		Cephalosporins & Antibiotics  & \emph{C.~acremonium}\\
		Statins & Cholesterol-lowering drugs & \emph{P.~citrinum}, \emph{M.~ruber} \\
		Cyclosporin A & Treatment of organ-transplant patients & \emph{T.~inflatum} \\
		Gibberellins & Fruit cultivation & \emph{G.~fujikuroi} \\
		Lactones, peptides, terpenoids & Food industry (flavourings) & \emph{T. viride}, \emph{G. fujikuroi}, \emph{M. circinelloides}, \emph{P. blaksleeanus}\\
		Heterologous proteins & Healthcare industry & \emph{A. niger}, \emph{A. oryzae}, \emph{A. nidulans}, \emph{T. reesei}\\
		\bottomrule
	\end{tabularx}
\end{table}

The exploitation of fungi by man is by no means a recent phenomenon, with processes such as the fermentation of alcoholic beverages dating back to ancient times. Some authors have suggested that the beginnings of biotechnology were coincident with the dawn of agriculture; the advent of large-scale grain production and the domestication of cows and goats was probably quickly followed by early forays into the production of alcoholic beverages and fermented milk products \cite{enari1999}. Many traditional Japanese foods, such as soy sauce and miso, have been produced in fermentation processes for thousands of years \cite{gomi2008} and the Japanese biotechnology industry has used this vast experience to establish itself as a world-leader in the production of fungal metabolites. Fungi have also been traditionally utilised as sources of food themselves, as is the case with mushrooms (\emph{Agaricus bisporus}, for example). More recently, \emph{Fusarium venenatum} has been developed as single-cell protein for human consumption under the trade name of Quorn\texttrademark, the protein of which is comparable in nutritional quality to that of chicken. However, the conversion of \lq feedstock' into protein is more efficiently achieved by \emph{F. venenatum} than by chickens \cite{glazer2007}. A chicken gains approximately 49~g of protein per kilogram of protein-containing feed consumed and the animal doubles in weight in 3~weeks. However, \emph{F. venenatum} requires just a simple feed consisting of glucose and some inorganic salts, each kg of which is converted into 136~g of protein, with biomass doubling every 6~hours!

Up to the early 1980's, the majority of products derived from microbial fermentation were for use in the food industry, making up an estimated 80\% of the market value of biotechnology applications in 1981 \cite{glazer2007}. However, since the use of microbes for the production of therapeutic compounds was clinically approved in 1982 (initially for the production of insulin by \emph{Escherichia coli}), successful attempts have been made to exploit fungi as expression hosts of heterologous proteins \cite{iwashita2002}, due to their capacity for high levels of excretion and the GRAS status\footnote{\lq Generally regarded as safe'; US Food and Drug Administration} obtained by many (such as \emph{A. oryzae}). This has resulted in many compounds of medicinal value, previously only obtainable from mammals, being produced using microbes. Diabetics, for example, were previously prescribed insulin derived from animal pancreatic tissue, which occasionally provoked immune responses in patients, but also carried a risk of infection from harmful pathogens \cite{kreuzer2005}. Recombinant microbes have also been used to produce factor IX, a deficiency of which causes hemophilia B, obviating the requirement for blood transfusions and the associated risks of infection \cite{kreuzer2005}. The production of these proteins was initially conducted using \emph{E. coli}, but the intra-cellular accumulation of the proteins of interest by the bacterium requires the use of expensive extraction techniques, which result in the inactivation of a portion of the product \cite{archer2001}. Filamentous microbes, however, secrete enzymes prodigiously in their natural environment and, as such, they are a more suitable choice of host organism for such processes. Research into microbial development of vaccines against human diseases such as hepatitis, influenza, rabies and HIV is also underway \cite{glazer2007}.

\subsection{Submerged industrial processes}

Production facilities typically employ \lq trains' of bioreactors ranging from 20 to 250,000~L (or even larger for some processes) \cite{chisti2001}, although it is desirable to keep the number of reactors to a minimum, as the costs associated with a reactor increase in proportion to $(\mbox{reactor volume})^{0.7-0.8}$ \cite{kristiansen2001}. Culturing begins in the smallest vessel and at a predetermined point in time during the exponential growth phase, the contents are used to inoculate the next fermenter in the chain (each vessel is usually approximately ten times larger than its predecessor). When the biomass level has increased sufficiently, the culture is transferred to the next fermenter in the chain, continuing in successive stages until the largest vessel (the production fermenter) is reached. This inoculum \lq work-up' avoids the long lag phase that would otherwise result from inoculating a large vessel with a relatively low level of biomass. Furthermore, inoculating with exponentially-growing cultures minimises any lag in growth at each stage in the chain. At the end of the process, the biomass is separated from the medium and the metabolite of interest is extracted. Industrial fermentations are commonly conducted on a large scale in a stirred-tank bioreactor, although airlift and bubble column reactors are also used extensively \cite{chisti2001}.

The bioreactor was first introduced in the 1940's for the large-scale production of penicillin from \emph{P. chrysogenum} (known at the time as \emph{P. notatum}) \cite{carlile2001} and the basic design remains unaltered, although monitoring equipment has increased in sophistication. In addition to the those components shown in Figure~\ref{fig:BioReactor}, reactors are also fitted with side ports for pH, temperature and dO\sb{2} probes as minimum requirements, in addition to connections above the liquid level for acid and alkali addition (for pH control), antifoam addition and inoculation. A foam breaker or foam rake will sometimes be fitted in the headspace above the liquid for processes where the use of antifoam is unacceptable. The exterior of the vessel is typically fitted with a water jacket to dissipate the heat of metabolism and that generated from culture agitation. Temperature control in this manner can be problematic in large vessels, as surface area for heat transfer decreases with increasing volume, and, as such, larger vessels will often be equipped with internal heat exchanger coils.

Most industrial fermentation processes are aerobic, air typically being introduced via a sparger with small holes below the impeller shaft. This facilitates the production of small air bubbles whose presence in the medium is prolonged as a result of the turbulent flow created by the impellers, aiding convective mixing and maximising their residence time. The combination of small bubbles (large total surface area) and long residence time maximises the diffusion of oxygen from the bubble into the medium and carbon dioxide from the medium into the bubble. Introducing air at the foot of the vessel also takes advantage of a relatively high hydrostatic pressure, which reduces bubble volume, therefore increasing surface-to-volume ratio, and maximising diffusion. Maintaining a positive pressure within the bioreactor can also aid diffusion.

\begin{figure}[htbp]
	\centering
	\pstool[width=10.5cm]{../C1/BioReactor}{
		\psfrag{S}[Bl]{\al Sparger}
		\psfrag{A}[Bl]{\al Sterile Air In}
		\psfrag{I}[Bl]{\al \hspace{0.25cm} Impeller}
		\psfrag{B}[Br]{\al Baffle \hspace{-0.3cm}}
		\psfrag{O}[Bl]{\al Air Out}
		\psfrag{D}[Bl]{\al \hspace{0.1cm} Impeller Drive Shaft}}
	\caption{Schematic of a typical stirred-tank bioreactor. Four baffles are usually used to prevent vortexing, with the baffle width ranging from $D/10$ to $D/12$, where $D$ is the vessel diameter. The number of impellers used is determined by the aspect ratio of the vessel, which is typically $\sim 3.5$. The lower-most impeller is located $\sim D /3$ from the bottom of the reactor, with additional impellers spaced $\sim 1.2$ impeller diameters ($d$) apart, where $d \approx D/3$ (dimensions from \cite{chisti2001}).}
	\label{fig:BioReactor}
\end{figure}

Mechanical agitation, which also aids in the convective mixing of nutrients present in the media, is typically provided by Rushton impellers mounted on a central vertical shaft\footnote{A Rushton impeller consists of six to twelve flat, rectangular plates mounted perpendicularly on a horizontal disc, with the plates' horizontal alignment being radially coincident with the disc.}. Together with medium sterilisation, culture mixing is one of the most energy-intensive aspects of fermentation, but the power requirement for agitation depends greatly upon the rheological properties of the broth, which in turn depends upon the growth form of the cultivated organism; pelleted structures will typically result in a Newtonian fluid\footnote{A Newtonian fluid is characterised by a linear relationship between stress and strain rate, the constant of proportionality being the viscosity. The flow properties of a non-Newtonian fluid are not described by a single constant value of viscosity.}, but filamentous growth often results in a pseudo-plastic, non-Newtonian fluid (viscosity decreases with increasing shear). For some processes, in which the cost of power required for mechanical agitation is deemed to be prohibitive, \lq tower' fermenters are employed, in which agitation is provided by supply of compressed air. Agitation can also have a significant impact on the morphology of the organism (Section~\ref{sec:agit}).

Other significant costs associated with fermentation processes are raw materials for use as substrate and transport of same. In a laboratory environment, medium composition is typically strictly defined with the use of pure chemicals. However, on an industrial scale, this is not practical as the volume of media required is several orders of magnitude greater than that used in a bench-top fermenter. A cheap, carbon-rich feedstock must therefore be sourced and quite often, waste products from other industries, such as molasses (waste liquor from sugar refineries), are well-suited. The costs associated with transporting large quantities of these raw materials can often be considerable and, as a result, the fermentation plant may be sited near the source of raw materials. Medium ingredients alone can represent up to 60\% of operating costs, with the carbon source alone constituting up to 90\% of raw material expenditure \cite{kristiansen2001}.

\subsection{Solid-state processes}

Solid-state fermentation (SSF) involves the cultivation of microbes on solid particles, typically under conditions of low water activity. The majority of SSF processes are aerobic, involve filamentous fungi (although some involve bacteria and yeasts), and may be classified as either natural (indigenous), such as ensiling or composting, or pure culture, using individual or mixed strains \cite{mitchell2006,pandey2008}.

The success of submerged culturing (SmF) during the 20\sp{th} century has led to a decline in the use of SSF. However, the practice is still ongoing and has shown promise for the production of many enzymes, acids and bioactive compounds, leading to increased research interest \cite{pandey2000,mitchell2006}. However, due to the different physical nature of SSF compared to SmF (particularly, the presence of solid-air interfaces in SSF), hyphal extension and branching patterns in mycelial organisms may be quite different between the two systems \cite{mitchell2006}. Recent studies have indicated differential protein expression in solid culture by several organisms \cite{tebiesebeke2002,iwashita2002}, exemplifying future potential application of this fermentation format. There is also evidence that solid-state cultivation can result in higher biomass yields and higher enzyme titres compared to submerged fermentation \cite{smith1983,machida2002}, possibly due to the closer resemblance to the natural environment of the microbes \cite{oda2006}. Furthermore, different products may be produced in SSF. For example, Ruijter and colleagues found that \emph{A. oryzae} accumulates polyols (glycerol, erythritol and arabitol) at low water activity on a solid substrate \cite{ruijter2004}. It was suggested that this might be typical for SSF due to the specific growth conditions present during growth on a solid substrate. However, much of this investigative work remains confined to laboratories; the potential of SSF to operate reliably at a large scale has not been investigated to the same degree as the SmF format \cite{pandey2000}. However, examples of production-scale SSF facilities do exist, such as that operated by Alltech\footnote{\href{http://www.alltech.com}{http://www.alltech.com}} in Serdan, Mexico, claimed to be the first in enzyme technology based on SSF.

Traditionally associated with the production of oriental foods, the use of solid media has certain merits, such as low energy requirements, abundance of cheap raw materials and a low water requirement \cite{smith1983}. Agro-industrial crop waste and residues are typically utilised as substrates and, while wheat bran has proved to be the most popular choice, a variety of other substances have been investigated, such as cassava, soya bean, sugar beet, potato, crop residues and residues of the fruit-processing, coffee-processing and oil-processing industries (reviewed extensively by Pandey and colleagues \cite{pandey2000}). However, the substrate may require chemical or mechanical pre-treatment prior to fermentation \cite{pandey1999} and the sterilisation of a solid substrate is typically more difficult to achieve than a liquid medium. However, SSF often involves an organism capable of tolerating low water activity, and, if an active inoculum is added to a (cooked) substrate, the process organism is able to out-compete potential contaminating organisms, meaning that strict aseptic operation of the bioreactor may not be essential \cite{pandey2000}.

However, SSF is often slower, is difficult to control due to the lack of suitable sensors and probes and the dissipation of metabolic heat is often problematic \cite{smith1983}. Environmental parameters in SSF, such as temperature, pH, concentrations of oxygen and nutrients, porosity and other physical properties of the solid matrix as well as fungal biomass are difficult or even impossible to measure online due to the lack of free liquid, the complexity and heterogeneity of the solid material as well as the intimate interactions between the microorganisms and their substrates \cite{lenz2004}. Furthermore, there may be severe restrictions in the supply of O\sb{2} to a significant proportion of the biomass, large nutrient concentration gradients can exist within solid particles and the movement of solid particles can cause impact and shear damage to hyphae \cite{mitchell2006}. SSF is also generally more labour-intensive, although a fully automated system for soy sauce production has been developed in Japan \cite{machida2002} and several commercial enzymes are produced using solid substrates \cite{oda2006}. However, while a variety of computational and modelling methods have been applied to the study of solid-state fermentations, experimental data on this culture format is still lacking \cite{lenz2004}.