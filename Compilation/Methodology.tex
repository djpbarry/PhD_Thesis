\chapter{General Materials \& Methods}\label{ch:MaterialsMethods}

\section{Preparation of spore inoculum}\label{sec:InocPrep}

Conidial suspensions were prepared in one of two ways. In the first method, standardised \emph{Aspergillus oryzae} (ATTC 12891) conidial suspensions were prepared from malt agar (Lab M LAB037) slant cultures (after incubation for 7~days at \celc{25}) by the addition of 5~ml phosphate-buffered saline (PBS) containing Tween 80 (0.1\% v/v; PBS-Tween-80). The cultures were briefly vortexed and the suspensions recovered by aspiration with a Pasteur pipette. Conidium concentration and the absence of hyphal elements were assessed using a Neubauer chamber, and the conidia were subsequently pelleted by centrifugation at 3,000~rpm for 30~min at \celc{4}. The pellet was then re-suspended in PBS-Tween-80 and glycerol (20\% v/v, final concentration) to yield \inoc{1}{6} and stored at \celc{-20}.

The second method is similar to that described by O'Cleirigh and colleagues \cite{ocleirigh2003}. \emph{A. oryzae} was grown on malt agar in 250~ml Erlenmeyer flasks for 7 days at \celc{25}. A spore suspension was prepared by washing the surface of the culture with 25~ml (PBS-Tween-80) containing approximately 3~g glass beads (BDH) and agitating at 120~rpm, \celc{25} for 30 minutes. The suspension was filtered through sterile glass wool to remove hyphae and the conidia were subsequently pelleted by centrifugation at 3,000~rpm for 20 minutes at \celc{4}. Conidium concentration was standardized using a Neubauer chamber to yield a stock concentration of \inoc{1.8}{8} and aliquots were stored in glycerol (20\% v/v) at \celc{-20}.

\subsection{Assessment of spore viability}

The viability of spores after freezing was determined using the pour plate method \cite{thatcher1968}. Vials containing spore suspension were thawed from frozen at \celc{37} and serially diluted with sterile PBS-Tween-80 to yield suspensions of approximately $5 \times 10^3$, $1 \times 10^3$ and \inoc{1}{2}. An aliquot (1~ml) of each spore concentration was transferred to the centre of a separate sterile Petri dish and immersed in approximately 25~ml of molten malt agar (\celc{40}). Each spore concentration was plated in triplicate. The dish was gently swirled to disperse the spores and allowed to cool at room temperature for 1~hour. Once the agar had set, the dishes were incubated at \celc{25} for approximately 36~hours and any visible colonies were manually counted. The percentage mean spore viability ($v$) was calculated according to:

\begin{equation}
	v = \frac{100}{C} \sum \frac{c_i}{3d_i}
\end{equation}

\noindent where $C$ is the nominal spore concentration in the frozen stock, $c_i$ is the mean colony count for each plated spore concentration and $d_i$ is the corresponding dilution factor for that concentration ($1/10$, $1/100$, etc.). The resultant values are referred to in individual experiments.

\section{Preparation of buffers}

Phosphate-buffered saline was routinely prepared by dissolving one tablet (Oxoid Dulbecco \lq A' BR0014) in 100~ml of distilled water. Where specified, Tween-80 was added at a concentration of 0.1\%~(v/v). All conidium dilutions used in the experiments described were performed in sterile PBS-Tween-80.

\section{Basal medium for microorganism cultivation}\label{sec:BasalMedium}

The basal medium used for both submerged and solid state fermentation of \emph{A. oryzae} was a modification of that described  by Amanullah and colleagues \cite{amanullah2000}: Citric Acid, \gl{2.0}; MgSO\sb{4}.7H\sb{2}O, \gl{2.0}; KH\sb{2}PO\sb{4}, \gl{2.0}; (NH\sb{4})\sb{2}SO\sb{4}, \gl{3.0}; CaCl\sb{2}.2H\sb{2}O, \gl{1.1}; K\sb{2}SO\sb{4}, \gl{2.0}. A trace metal solution was added (0.5~ml/L), consisting of: Citric acid, \gl{3.0}; ZnSO\sb{4}.7H\sb{2}O, \gl{0.5}; FeSO\sb{4}.7H\sb{2}O, \gl{0.5}; CuSO\sb{4}, \gl{0.25}; MnSO\sb{4}.H\sb{2}O, \gl{0.28}; NiCl\sb{2}.6H\sb{2}O, \gl{0.09}. Media pH was adjusted using either 2~M HCl or NaOH as required using an electronic pH metre (Hanna Instruments 8519 or pH210) calibrated against standard buffer solutions (AVB Titrinorm 32044.268 \& 32045.262). All media were sterilised by autoclaving at \celc{121} and 1~atm for 15 minutes.

\section{Optimised protocol for membrane immobilisation of culture and subsequent visualisation}\label{sec:OptAssay}

\subsection{Cell immobilisation and solid-state cultivation}

The basic conidiospore immobilisation procedure consisted of filtering a suspension (\ml{25} of approximately \sm{400}) through a cellulose nitrate membrane (Sartorius Stedim 11306-47-ACN, Millipore HAWG 047 SO or Pall 66278) using a membrane filtration device (Millipore) connected to a vacuum pump. A spore concentration of this magnitude was found by trial-and-error to be optimum in preventing over-crowding of mycelium on the membrane upon germination, while suspension volumes in excess of 20~ml were adequate to ensure a uniform spore coverage of the membrane. After washing with PBS-Tween-80 (for removal of any wall-adherent cells) and sterile water (removal of excess PBS-Tween-80), the membrane was overlaid evenly on to the surface of malt agar and incubated at \celc{25}.

\subsection{Processing of culture for image analysis}

In the optimised procedure derived from the experiments described in subsequent chapters, the membrane was removed from the agar after a suitable period of time and replaced in a filtration device where it was exposed to fixative solution (phenol 20\% w/v, glycerol 40\% v/v, lactic acid 20\% v/v in distilled water) for 5~minutes. Following washing with PBS-Tween-80 and then water, the membrane was placed in a Petri dish and dried at \celc{65} (75~min). After staining with lacto-phenol cotton blue, the membrane was rinsed with PBS-Tween-80 (5~min) and distilled water, followed by cutting and mounting on a microscope slide and drying at \celc{65} (75~min). The membrane was rendered transparent and suitable for imaging by treatment with microscopy immersion oil (Olympus AX 9602).

\subsection{Microscopic visualisation of submerged culture}

Samples (1~ml) were taken from shake-flasks and added to approximately 25~ml PBS-Tween-80 before filtering through a cellulose nitrate membrane using a filtration device. The membrane was then fixed, stained and dried as described above.

For fluorescence microscopy, samples were diluted approximately as necessary with PBS-Tween-80. The diluted sample (1~ml) was added to 4~ml of calcofluor white (0.01\% w/v; Sigma F3543) and incubated at room temperature for 10~minutes, before filtration through a cellulose nitrate membrane and drying at \celc{65} (75~minutes).

\section{Microscopy and image capture}\label{sec:microscopy}

Light microscopy was performed with a Leica DMLS2 microscope, with the condenser and aperture fully opened and maximum illumination. The microscope is fitted with a $\times 10$ eyepiece lens, four standard objective lenses ($\times 4$, $\times 10$, $\times 20$ and $\times 40$) and a single oil-immersion objective ($\times 100$). All lenses, filters and light-sources were cleaned thoroughly with lens tissue prior to use. Standard glass microscope slides were used in all experiments.

All images were captured using a Canon Powershot S50 camera attached directly to the microscope. The camera could be controlled remotely from a PC using Canon's \emph{RemoteCapture} utility. The camera lens was cleaned with lens tissue prior to use. Images were stored in JPEG format with minimal image compression. The image resolution used is referred to in individual experiments.

Inter-pixel distances were calibrated by imaging a stage graticule, from which a standard scale factor (\mic{0.138}~pixel\sp{-1} for $2592 \times 1944$ image captured at $\times 400$) was calculated that was subsequently used in all experiments.

Images of fluorescent samples were captured with a Canon PowerShot S50 digital camera attached to a fluorescence microscope (Leitz Laborlux S) fitted with an epifluorescence illuminator (307-148.002 514687, Leitz Wetzlar). Images were captured at $\times 100$ magnification.

\subsection{Visualisation of fungal macro-morphology}\label{sec:PelletScan}

Fungal macro-morphology was visualised using a method similar to that described by O'Cleirigh and colleagues \cite{ocleirigh2003}. Samples (approximately 10~ml) were decanted from shake flasks and centrifuged at 1,000~rpm, \celc{4} for 35 minutes. Supernatant was removed and the biomass was immersed in lactophenol cotton blue with gentle agitation to ensure stain uptake. Distilled water was added to a volume of 30~ml and the contents gently agitated before being subjected to centrifugation (1,000~rpm, \celc{4} for 20 minutes). The supernatant was replaced with distilled water before further centrifugation. This process was repeated until excess stain was removed and the supernatant became clear. The stained biomass was then transferred to a Petri dish and diluted to yield a coverage of approximately 1~cm\sp{2} biomass per 20~cm\sp{2}. The contents of the Petri dish were then scanned with a Hewlett Packard Scanjet G4010 flatbed scanner, with the images stored as 24-bit bitmaps (300~dpi).