\phantomsection
\addcontentsline{toc}{chapter}{Abstract}
\section*{Abstract\markright{ABSTRACT}{}}\thispagestyle{plain}

Mycelial morphology is a critically important process property in fermentations of filamentous micro-organisms, as particular phenotypes are associated with maximum productivity. The design of systems capable of rapidly and accurately characterising morphology within a given process represents a significant challenge to biotechnologists, as the complex phenotypes that are manifested are often not easily quantified.

A system has been developed for high-resolution characterisation of filamentous fungal growth, using membrane immobilization and fully-automatic image-processing software. The system has been used to quantify the early-stage hyphal differentiation of \emph{Aspergillus oryzae} on solid substrates, by measuring spore projected area and circularity, the total length of a hyphal element, the number of tips per element, and the hyphal growth unit. Spore swelling expressed as an increase in mean equivalent spore diameter was found to be approximately linear with time. Widespread germination of spores was observed by 8~h after inoculation. From approximately 16~h, the number of tips was found to increase exponentially. The specific growth rate, maximum hyphal tip extension rate and specific branching frequency of a population of hyphae were calculated as approximately \h{0.27}, \h{\mic{27}~tip\sp{-1}} and \h{\mic{$ 2.3 \times 10^{-3}$~tips}\sp{-1}} respectively. The robustness of the image-analysis system was verified by testing the effect of small variations in the input parameters.

Subsequent experimentation focussed on investigating the morphological development of \emph{A. oryzae} in submerged culture and the associated influence on $\alpha$-amylase production. The temporal variation in pellet structure and $\alpha$-amylase production over time was quantified and the potential for the use of membrane-immobilisation in submerged culture was examined. Variation in carbon source type had little morphological impact, although increasing starch concentration caused a shift from a pelleted form to dispersed, \lq pulp-like' growth. Increasing inoculum concentration was found to result in a decrease in mean pellet diameter and an increase in $\alpha$-amylase production. The supplementation of fermentation media with non-ionic detergents caused a significant increase in $\alpha$-amylase production (up to 149\%), but this increase did not seem to be related to observed morphological variation.

Recently, fractal geometry has been employed in the study of filamentous microbes, but a clear link between fractal dimension and branching behaviour has not been demonstrated. This thesis presents an alternative means of enumerating the fractal dimension of fungal mycelial structures, by generating a \lq fractal signal' from an object boundary. In the analysis of a population of \emph{A. oryzae} mycelia, both fractal dimension and hyphal growth unit were found to increase together over time, while cultivating populations of \emph{Penicillium chrysogenum} and \emph{A. oryzae} mycelia under a variety of different conditions revealed a strong correlation between fractal dimension and hyphal growth unit. The technique has the potential to be adapted and applied to any morphological form that may be encountered in a fermentation process, providing a universally-applicable parameter for more complete data acquisition.