\section{Discussion}\label{sec:DevImageAnalDisc}

The accurate characterisation of fungal morphology remains a key target for industrial biotechnologists and the deployment of image processing systems is central to achieving this goal. While extensive progress has been reported in the use of computer-aided analysis of mycelial structures \cite{cox1998,papagiannireview}, many implementations of image analysis systems described in the literature depend on significant manual intervention for accurate quantification to be achieved \cite{mcintyre1997,mcintyre1998, wongwicharn1999,wongwicharn1999a,lubbehusen2003,lubbehusen2004,anikster2005,rahardjo2005b,lecault2007,elsabbagh2008}.  Hence there still exists a need for further development of automated systems for the classification and analysis of filamentous microbial conformations.

In this study a new system has been presented for quantitative analysis of both micro- and macro-morphology. The method obviates the need for purchase of relatively expensive commercial software by adding utility to the publicly available ImageJ platform \cite{imagej}, a system of proven usefulness in this field \cite{rahardjo2005b,papagianni2006b}. The prioritisation of speed of execution has yielded a system capable of both rapid and accurate analysis of a large bank of images, permitting near-\lq real time' assessment of fungal development through the compilation of population data. While the on-line examination of individual hyphal elements has yielded valuable data on apical extension \cite{spohr1998}, population statistics are essential for the characterisation of bioprocesses, due to the inherent variations in growth kinetics throughout a given population \cite{paul1993,spohr1998}.

In common with all image-analysis systems, the routines described here depend on high-quality input images and are sensitive to elevated levels of artifact, low-level or uneven sample illumination and out-of-focus fields of view.  Uniform staining of hyphal elements is also essential to ensure against the incorrect classification of mycelial structures. Contiguous, high-contrast staining also permits automatic image segmentation via the application of grey-level thresholding, obviating the requirement for manual segmentation \cite{mcintyre2001} or the specification of a fixed threshold value to be used for all images \cite{hitchcock1996}. While the image pre-processing described alleviates these concerns to some degree, high-quality sample preparation will ensure optimal system performance and the generation of accurate morphological data. The dilution of biomass prior to image capture may also be necessary so as to minimise hyphal overlapping and contact between spores or pellets.

The system described here has much in common with those previously described in the literature. The use of simple morphological thresholds to exclude artifacts is a commonly adopted approach \cite{lucumi2005,lecault2007}. This simple technique is generally effective as artifacts tend to be small relative to hyphae and also exhibit a higher circularity index ($C$). The use of morphological filters to remove small breaks in objects has previously been utilised, although caution was advised by Lucumi and colleagues against multiple iterations, which may result in the removal of \lq gaps' in mycelial networks \cite{lucumi2005}. Such an approach was refined by Daniel and colleagues to join small gaps in skeletal structures \cite{daniel1995}.

\emph{Skeletonisation} has also been widely utilised in the study of hyphal elements \cite{tucker1992,spohr1998,denser-pamboukian2002,rahardjo2005b,slaba2005,bizukojc2006,lecault2007}, as it facilitates rapid enumeration of hyphal lengths and tip numbers. However, such an approach is not without short-comings and, as was the case in this study, the specification of \emph{pruning} algorithms is often necessary \cite{drouin1997,spohr1998,denser-pamboukian2002,lucumi2005,lecault2007}, although in some cases this is due to the presence of solid particles in the media and a need to ensure such artifacts are not interpreted as small branches. Another dimension was added to the analysis of skeletal structures by Drouin and colleagues, as the use of the binary skeleton as a mask allowed the precise quantification of \lq full' and \lq empty' hyphal zones, together with septation, in filaments of \emph{Streptomyces ambofaciens} \cite{drouin1997}. Such an approach could potentially be utilised to delineate \lq active' and \lq non-active' regions in calcofluor white-stained preparations \cite{amanullah2002}. Other authors have avoided using \emph{skeletonisation} algorithms as small changes in hyphal boundaries can have a significant impact in the performance of the algorithm \cite{sonka1993}. This is particularly true when swollen hyphal tips are the object of study, but the omission of such a thinning step has generally resulted in the necessity for manual tip counts \cite{muller2003}, resulting in a considerable increase in processing time. In the absence of \emph{skeletonisation}, the automatic detection of hyphal tips may be implemented using either the application of morphological filters to the binary mask of the mycelium, or an interpretation of the mycelium boundary, in which tips would manifest themselves as sharp turning points.

The production of a mycelial graph could be potentially applied to the study of a microbe's foraging strategy on a solid substrate or the response to changes in environmental conditions. A similar \lq segmental' approach to mycelial characterisation was previously described in the analysis of \emph{Trichoderma viride} colonies \cite{hitchcock1996}. Such an approach permits an estimation of the order in which branches were formed on \lq free' elements, previously determined by iterative erosion \cite{tucker1992,spohr1998,lecault2007}. The inclusion of hyphal width in such a graph would add further utility and permit the modelling of the mycelium as a \lq fungal network' \cite{fricker2008}. However, the estimation of hyphal width from images in which the hyphae are perhaps 2 -- 3 pixels in width (which is sensitive to the grey-level threshold) is difficult, considering the width can only be measured to within one pixel. The average width could be estimated by dividing the projected area by the total length of the structure, but this is of little value in the context of a network. An alternative means of estimating hyphal width was proposed by Hitchcock and colleagues, which involved convolution with a Gaussian point-spread function and subsequently summing over the resultant grey levels based on the location of the hyphae in the binary mask \cite{hitchcock1996}.

More extensive characterisation of pellets has been described in other reports. M\"{u}ller and colleagues estimated the \lq compactness' of pellet structures by calculating the ratio of projected area to projected convex area, which was defined as the projected area after filling any internal voids and any concavities in the external perimeter \cite{muller2003}. Papagianni and Mattey quantified pellets in terms of their eccentricity, a measure of the degree to which a shape deviates from a circle \cite{papagianni2006a}. Other generally-applicable measures include the dimensions of the object's bounding box and the convex perimeter \cite{lecault2007}. A more thorough assessment of pelleted morphologies is possible with the system described here. However, it has been assumed here that images of pellets to be analysed are captured macroscopically, using either a camera or, preferably, a flatbed scanner. For more extensive examinations and the derivation of data beyond the projected area, it is recommended that microscopic imaging be utilised, permitting the isolation of individual hyphae and the accurate localisation of the pellet perimeter.

As with pellets, an analysis of mycelial aggregates is also possible with the system described here, using parameters such as projected area and circularity. Other parameters, such as those mentioned above, could easily be incorporated if necessary. Lucumi and colleagues described a useful algorithm for the extensive quantification of such aggregates, which involved the object's separation into core and filamentous fractions and a subsequent estimate of hyphal growth unit based on the number of tips found \cite{lucumi2005}. A common measure that has been applied to the dispersed growth form is the \lq main' hyphal length \cite{tucker1992}, which Wongwicharn and colleagues described simply as the tip-to-tip length of the longest hyphal length in a mycelium \cite{wongwicharn1999a}. While the identification of the main hypha is trivial for free elements (no hyphal \lq crossovers'), the utility of such a measure in the assessment of mycelial aggregates is limited. The use of more universally applicable measures, such as projected area, is more appropriate; projected area was described by Li and colleagues as being particularly useful for the determination of fragmentation in a bioreactor \cite{li2002}. Projected area has also been demonstrated to correlate well with measures of dry-cell weight in submerged culture, due to the relative constant width of hyphae in these processes \cite{carlsen1996a,spohr1997}. Measures such as projected area have the added advantage of obviating the need to discriminate between different morphological forms, a capability that has been described in detail in some studies \cite{tucker1992,papagianni2006a}. However, the information that may be garnered from such simple measures is obviously extremely limited and, given the evidence for metabolite excretion occurring primarily at hyphal tips \cite{wosten1991,amanullah2002,muller2002}, an accurate evaluation of branching is vital in studies of filamentous microbial development. 

The image-processing system does have some difficulty in distinguishing between \lq young' hyphae (recently germinated spores) and spore clusters, because of their morphological similarity in terms of projected area and circularity (in the approximate range 0.4 -- 0.8). As such, these young hyphae and spore clusters are typically excluded from the results. In order to fully characterise this transition from spore to hypha (as would be necessary in studies of germinative potential, for example), a means of distinguishing between these different objects is required. Boundary shape descriptors \cite{wilson2002,pazoti2005} could be utilised in conjunction with the morphological thresholds used in the current study to this end. A means of breaking up clumps of spores in the inoculum, such as sonication, may also help to alleviate the problem. However, restrictions posed by such limitations should be minimal, as the development of spores and the development of hyphae are typically subject to separate analyses, due to the different nature of their respective growth mechanisms (swelling versus polarised tip extension). It should however be noted that in the analysis of spore morphology, the use of \emph{watershed} techniques to separate touching objects is not recommended for images containing germinated spores, as it is likely that the germ tubes will be segmented from the parent spore and both objects interpreted as separate objects.

\section{Conclusions}

Image analysis systems represent essential enabling tools for the accurate quantification of filamentous microbial morphology. Despite the significant progress that has been achieved in producing such systems, a requirement for manual intervention is still commonplace and, as such, further development, with an emphasis on process automation, is necessary. The system described here is capable of rapidly and accurately producing population statistics derived from a large bank of images of either fungal spores or mycelia at the microscopic level, or pellets at the macroscopic level. Such data can then be utilised to derive information on the growth kinetics, by presenting multiple image banks representing different stages in the organism's development.

The parameters used in quantifying morphology in this study (projected area, circularity, total hyphal length and number of hyphal tips) are similar to those utilised in other studies of filamentous microbes. However, other measures (eccentricity, convex area, bounding rectangle) can be easily incorporated if necessary. The analysis of mycelia is more extensive than that typically reported in other studies and allows a more complete characterisation of the microbe beyond the conventional hyphal growth unit. The basic utility of the system has been demonstrated, but more extensive application will be detailed in the subsequent chapters of this thesis, as image processing is employed in the derivation of growth kinetics on solid substrates and development in submerged systems.