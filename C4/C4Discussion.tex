\section{Discussion}

The ability to derive kinetic parameters to describe industrial fermentations is a powerful tool in facilitating process optimisation. A link between microscopic and macroscopic kinetics has long been established \cite{emerson1950, marshall1960}, but more recently, detailed examinations of growth kinetics have been linked to specific macro-morphological forms \cite{muller2002,muller2003,eypark2006}. Relationships between micro-kinetics and metabolic activity have also been identified, suggesting that knowledge of microscopic parameters is crucial to furthering an understanding of metabolite production \cite{carlsen1996a, agger1998, papagianni1999, papagianni2001}. Indeed, many of these dependencies have been expressed in mathematical form and combined to yield model simulations, which have proved extremely useful in describing the time-course of development \cite{agger1998, christiansen1999, jhkim2005, cross2004, li2000}.

While there have been numerous detailed studies of the growth kinetics of filamentous organisms reported in the literature, the majority have focused on the development of microbes in submerged culture \cite{carlsen1996a, spohr1997, agger1998, spohr1998, christiansen1999, muller2002, lubbehusen2003, bizukojc2006, eypark2006, pollack2008}. Examinations of growth on solid culture have tended to be restricted to low-resolution assessments of the rate of change of colony projected area \cite{cross2004, couri2006}, for example.  More detailed investigations of individual hyphae in solid culture have been limited to a very small number of hyphae or hyphal elements \cite{trinci1974,dieguez-uribeondo2005, rahardjo2005b}.

The same is also true of many \lq on-line' studies of microbial development in submerged culture using some form of a flow-through cell \cite{spohr1998, christiansen1999, muller2002, lubbehusen2003, eypark2006, pollack2008}. While these chambers offer the advantage of a homogeneous environment that can be easily modified and controlled, deriving population statistics from such a small number of examinations is difficult. Indeed, many of these studies have illustrated that growth parameters can vary from one hypha or hyphal element to the next. In the on-line study of a single element of \emph{A. oryzae}, for example, Christiansen and colleagues found that values for $\Kt$ can vary significantly between individual hyphae, even within a single mycelium; when apical branching occurred, it was observed that the tip extension rate decreased temporarily \cite{spohr1998, christiansen1999}. These variations in the growth rate of individual tips invariably lead to inter-mycelial variations in the specific growth rate \cite{spohr1998}, further evidence for which is presented here (Fig.~\ref{fig:ThlDist}). However, by averaging over a large population, an accurate estimate of growth rate may be derived, as demonstrated both here (Fig.~\ref{fig:ThlNtT}) and in studies of \emph{A. oryzae} in submerged culture \cite{carlsen1996a, spohr1997}.

\subsection{Kinetics of spore swelling}

The approximately linear increase in \emph{A. oryzae} spore diameter described here, together with an invariant spore circularity (Fig.~\ref{fig:ApsCsT}), is in general agreement with other reports \cite{spohr1998}. The increase in spore diameter is relatively modest compared to increases of \mic{2.0} in 6~hours for spores of \emph{A. niger} \cite{bizukojc2006} and approximately \mic{7.0} in \emph{Rhizopus oligosporus} sporangiospores \cite{medwid1984}. Spohr and colleagues reported an equivalent diameter of approximately \mic{8.6} for spores of \emph{A. oryzae} just prior to germination \cite{spohr1998}. However, in all of these cases, the examination was conducted in submerged culture. Little difference in spore swelling rates was observed between \emph{A. oryzae} spores on different substrates (Fig.~\ref{fig:TADsT}). It has previously been found that virtually no uptake of nutrients occurs during the swelling of \emph{A. niger} spores \cite{bizukojc2006}, which may explain the absence of any detectable influence in this study. However, a carbon source was deemed necessary for the commencement of swelling of \emph{R.~oligosporus} sporangiospores while both carbon and nitrogen were required for germination - the spores did not contain sufficient endogenous carbon to support either process \cite{medwid1984}.

The germination of spores of \emph{A. oryzae} took place 6 -- 8~hours post inoculation in this study, which is in approximate agreement with the germination time of 6~hours reported by Carlsen and colleagues \cite{carlsen1996a}. A germination time of just 4~hours has also been found for \emph{A.~oryzae} \cite{spohr1998}, while spores of other \emph{Aspergilli} can take up to 10~hours to produce germ tubes \cite{bizukojc2006}. In this study, glucose was found to prolong the spore swelling process compared to other substrates (Fig.~\ref{fig:TADsT}). Media composition has also been suggested to have a significant influence on the germination of \emph{Penicillium chrysogenum}, as did inoculum age \cite{paul1993}, while water activity and temperature were also demonstrated to have a considerable effect \cite{sautour2001}. Both pH and temperature were reported to have a substantial impact on the germination of \emph{R. oligosporus} \cite{medwid1984}.

\subsection{Kinetics of hyphal development}

The description of growth kinetics presented here utilised the same basic principles previously described in other population studies involving filamentous microbes \cite{carlsen1996a, spohr1997}. While the derived values of $\kt$, $\kb$ and $\Kt$ have no mechanistic basis, they are useful for comparing the behaviour of organisms under different environmental conditions. To the best of the author's knowledge, this is the first such analysis involving a large population of hyphal elements cultivated on a solid substrate. However, the artificial environment in which the organism was grown must be taken into consideration when comparing with kinetic parameters derived in other systems. Rahardjo and colleagues contended that the presence of membrane filters reduces the maximum respiration rate and, consequently, the growth rate of \emph{A. oryzae} \cite{rahardjo2004}. However, differences in activity were not observed until well in excess of 24~hours post-inoculation, indicating that oxygen limitations were unlikely to have been significant in this work. Furthermore, the similarity of kinetic parameters derived here to those reported in other studies involving \emph{A. oryzae} suggest that, while an influence of the membrane cannot be ruled out, it does not appear to have been significant.

The use of digital image processing to derive information on growth rates has been found to be accurate with respect to the conventional means of deriving kinetics from dry cell weight measures in submerged cultures of \emph{A. oryzae} \cite{carlsen1996a, spohr1997}. However, the morphological form adopted by the fungus can influence the growth rate derived from biomass measurements; pelleted biomass may result in a lower specific growth rate due to substrate limitation compared to filamentous growth \cite{carlsen1996a}. A caveat to the use of measures of hyphal length as a means of deriving kinetic data is that hyphal diameter is relatively constant throughout the population being analysed, as has been demonstrated in submerged batch fermentations of \emph{A. oryzae} \cite{carlsen1996a, spohr1997, li2000}. However, it appears that this only holds true for a constant specific growth rate. Hyphal diameter has been shown to be directly proportional to the specific growth rate \cite{agger1998}, while it was also noted that the growth of a glucoamylase producing strain of \emph{A. oryzae} starved of glucose was characterised by significant reduction in hyphal diameter and a reduced growth rate \cite{pollack2008}.

A variety of factors have been reported as significant in influencing the growth rate of filamentous microbes, but the most commonly employed regulator is substrate concentration. J{\o}rgensen and colleagues demonstrated a relationship between dilution rate and the specific growth rate of \emph{A. niger} in chemostat culture \cite{jorgensen2007}. Pollack and colleagues noted that the growth of \emph{A. oryzae} ceased when starved of glucose, before growth recommenced, fuelled by endogenous carbon, following a defined lag-time \cite{pollack2008}. Specific growth rate, specific branching rate and final tip extension rate of \emph{A. oryzae} have all been modelled using Monod kinetics with respect to glucose concentration \cite{spohr1998}, while M\"{u}ller and colleagues also noted increases in the specific growth and branching rates in the presence of increased glucose concentration \cite{muller2002}. 

While no discernible influence of media composition on specific growth rate was noted in this study, reports in the literature have demonstrated otherwise. Ali described a significant increase in specific growth rate resulting from the addition of kaolin in the submerged culturing of \emph{A. niger} \cite{ali2006}. The concentration of both carbon and nitrogen sources had a substantial impact on the growth kinetics of \emph{Mortierella alpine} and it was suggested that growth is inhibited at high nutrient concentrations \cite{eypark2002}. However, the specific growth rate of \emph{Mucor circinelloides} was found to be relatively independent of both carbon source type and concentration \cite{lubbehusen2003}, although McIntyre and colleagues had earlier documented the effect of media composition on the maximum specific growth rate of \emph{Mucor circinelloides} (syn. \emph{racemosus}) \cite{mcintyre2002}.

%In surveying other factors influencing the growth kinetics of filamentous microbes, Carlsen and colleagues reported the considerable effect of pH and temperature on the specific growth rate of \emph{A. oryzae}. Cross and Kenerley also found temperature to be a significant influence on growth rate in modelling the development of \emph{Trichoderma virens} on soil \cite{cross2004}.

\section{Conclusions}

Detailed analysis of the kinetics of filamentous microbial growth is essential for the understanding of a process time-course. Metabolite production is dependent on the specific rates of biomass and branch formation - understanding precisely how fungal biomass develops over time is therefore crucial for the purposes of optimisation and reproducibility. While a number of studies of the growth kinetics of filamentous microbes have been reported, the vast majority have focussed on the submerged culture format.

The analysis of \emph{A. oryzae} spores conducted here, the results of which were found to be reliable within a certain tolerance of the pre-selected circularity threshold, confirms earlier findings suggesting that the rate of increase in spore diameter during the swelling process is linear with respect to time. However, the possible distribution of swelling rates within a population, coupled with the prevalence of non-viable spores within the population, presents a considerable complication. Differences in germination times may also have some bearing on the result. Cultivating the spores on different media does not seem to have a significant influence on spore development, although spores did seem to take slightly longer to germinate on glucose.

Detailed information on the hyphal development of \emph{A.~oryzae} on a solid substrate was derived, which approximated previously reported data for this organism.  While these kinetic parameters and associated empirical expressions have no physiological basis, they are useful for comparing different environmental conditions, or for comparing with other organisms. The specific growth rate on a glucose-rich medium was slightly higher when compared to starch or malt agar, but the difference was not deemed to be statistically significant. The results produced by the image analysis system were tested and found to be stable over a range of input parameters.