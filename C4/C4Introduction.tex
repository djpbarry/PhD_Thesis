\section{Introduction}

An understanding of the kinetics of fungal development is fundamental to the rigorous design of fermentation processes, as the time-course of production is intrinsically linked to the rate and mode of growth. Perturbations to the rate at which biomass is produced, or alterations to the morphological form that this biomass takes, will have implications for productivity. Furthermore, there is emerging evidence suggesting a fundamental link between kinetic parameters of growth at the microscopic level and resultant macroscopic conformations \cite{muller2002,muller2003,eypark2006}. Hence, there is significant potential utility in the quantification of structural variation at the microscopic level.

\subsection{The temporal analysis of micro-morphological development}

The life-cycle of filamentous microbes generally commences as a single spore, which is typically a few microns in diameter. Upon activation, the spore begins to grow in size, with the increase in diameter generally being linear with respect to time \cite{spohr1998, carlile2001}. When a critical size is attained, a germ tube emerges, the tip of which accelerates away from the spore until a constant linear rate of elongation has been reached \cite{spohr1998}. When the germ tube reaches a certain length, a lateral branch is formed behind the advancing apex, which accelerates to the same constant rate of elongation \cite{carlile2001}. This process continues as long as conditions to support this linear rate of tip extension are met. Exponential growth is attained as a result of each branch producing further branches at a constant specific rate, all of which attain the same constant linear rate of growth, and a composite structure termed a \lq mycelium' develops. The increase in biomass ($x$) may thus be described as:

\begin{equation}
	\frac{dx}{dt} = \mu x
\end{equation}

\noindent where $\mu$ is described as the specific growth rate (\oh). The critical branch length that results in a new branch being spawned is referred to as the hyphal growth unit ($\hgu$), which is approximately equal to the total hyphal length ($\Lh$) divided by the total number of tips ($N$) of any given mycelium \cite{plomley1959}:

\begin{equation}
	\hgu = \frac{\Lh}{N}
\end{equation}

\noindent This value will tend to be approximately constant across all mycelia within a given population.

Considering that  mycelial growth involves the duplication of a constant hyphal growth unit, then $\Lh$ and $N$ must increase exponentially at approximately the same specific growth rate. This was first demonstrated experimentally by Trinci \cite{trinci1974}, who cultivated colonies on cellophane-covered solid media and obtained morphological measures from enlarged photographic prints. After an initial period of discontinuous branch production, exponential growth was observed, with measured specific growth rates varying from approximately \h{0.25} for \emph{A. nidulans} up to \h{0.60} for \emph{Mucor hiemalis}. Spohr and colleagues produced similar metrics in their study of \emph{A. oryzae} mycelia, cultivated submerged in a small \lq flow-through cell' \cite{spohr1998}. The principle advantage of such a system is the ability to monitor the development of individual elements, and even individual hyphae, from a single spore, up to a mycelium several millimetres in length. Similar systems have since been used to analyse the early growth stages of \emph{M. circinelloides} \cite{lubbehusen2003,lubbehusen2004} and various \emph{Mortierella} species \cite{eypark2002,eypark2006}. However, the continuous monitoring of individual mycelia, while providing valuable physiological data, does not permit the examination of large populations, which is essential for the statistical characterisation of fermentation processes.

% Links between the specific growth rate, as derived from microscopic measurements, and biomass development at the macroscopic level have long been established. Emerson proposed a \lq{}cube-root' relationship between biomass accumulation and fermentation time, based on the postulation that hyphae extend outward from pellets at a constant rate; hence, the cube root of mean pellet volume increases linearly with time (given that $r= \sqrt[3]{\frac{3}{4\pi}V}$) \cite{emerson1950} and this relationship was confirmed experimentally by Marshall and Alexander \cite{marshall1960}. Several studies have since demonstrated correlations between particular kinetic parameters, most commonly the specific growth rate, and metabolic activity \cite{carlsen1996a, agger1998, papagianni2001}, suggesting that a precise knowledge of the growth characteristics of an organism in a given process will lead to a more intimate knowledge of the kinetics of metabolite production.

Evidence exists for a link between micro-morphological development of filamentous microbes and macro-morphological form. For example, Park and colleagues concluded that species of \emph{Mortierella} exhibiting a high branch formation rate formed \lq pellet-like' aggregates with a distinct core, while species attributed with a low branch formation rate formed hyphal aggregates without cores \cite{eypark2006}. In addition, M\"{u}ller and colleagues found that a mutant strain of \emph{A. oryzae} that exhibited a greater degree of hyphal branching (with respect to a wild-type strain) was less likely to form large, inseparable clumps in submerged culture \cite{muller2002}. It was also suggested that the positioning of branches (apically or sub-apically) may affect the formation and dimension of macroscopic structures. It has also been demonstrated that biomass aggregation is directly related to the specific growth rate in the early stages of \emph{Aspergillus niger} submerged cultivation \cite{grimm2004, grimm2005a}.

Recently, digital image processing has been utilised as a means of elucidating kinetic data \cite{carlsen1996a, spohr1997, agger1998, spohr1998, christiansen1999, muller2002, lubbehusen2003, bizukojc2006, eypark2006, pollack2008, cross2004, couri2006, rahardjo2005b}. Conventionally, parameters such as the specific growth rate were calculated based on, for example, physical measurements of biomass content per unit volume of fermentation broth. However, it is possible to derive the same rate measurements from an analysis of digital images, whereby the projected area or length of these elements is considered to be proportional to the biomass present in the system. This obviously assumes that the density of all elements in the population is constant and that the diameter of all hyphae in the population is approximately equal. While there is evidence that individual hyphae approximate ellipsoids rather than cylinders \cite{dieguez-uribeondo2004} and that hyphal diameter appears to be proportional to the specific growth rate \cite{agger1998, pollack2008}, studies have demonstrated a close relationship between growth rates derived from image analysis and those derived from measures of dry cell weight \cite{carlsen1996a, spohr1997}.

% The utility of these kinetic relationships has been exploited in the design of mathematical models to describe both the growth of microorganisms and global process variables.
By employing image processing techniques, which permit the rapid quantification of fungal morphology, the growth kinetics of a large population of mycelia (such as that present within a bioreactor) may be accurately determined. Carlsen and colleagues successfully characterised the development of \emph{A. oryzae} in batch culture using such an approach, producing data such as the specific growth rate, the specific branching rate and the mean tip extension rate \cite{carlsen1996a}. Similar data was produced by Spohr and colleagues in their comparative assessment of different strains of \emph{A. oryzae} and it was suggested that a more densely branched strain may have been favourable for protein production \cite{spohr1997}. While particular kinetic parameters can vary from one individual hyphal element to the next \cite{spohr1998, christiansen1999, dieguez-uribeondo2005}, simple equations can be derived to describe the average kinetic properties of a large population of hyphal elements, thus permitting the kinetic analysis of fermentation processes involving filamentous microbes.

\subsection{Influences on growth kinetics}

There is evidence in the literature suggesting that the specific rates of development may be influenced by a variety of factors. These include, but are not limited to, pH \cite{carlsen1996a}, temperature \cite{carlsen1996a, cross2004}, available oxygen concentration \cite{rahardjo2005b} and agitation speed in submerged culture \cite{papagianni1999, papagianni2001}. The morphological form adopted by an organism in submerged culture can also affect the specific growth rate, due to substrate limitations in pellets \cite{carlsen1996a}. Variations in temperature, pH and water activity have also been reported as significant in determining rates of spore swelling and germination in \emph{Penicillium chrysogenum} \cite{sautour2001} and \emph{Rhizopus oligosporus} \cite{medwid1984}.

Variations in substrate concentration are one of the more common means of effecting a change in the growth kinetics of microbes. Available evidence suggests that in many organisms, the response of the growth rate to changes in carbon source concentration may be described using Monod kinetics, where a saturation constant, typically denoted $K_s$, is the rate controlling factor \cite{spohr1998}. A relationship between dilution rate and the specific growth rate of \emph{A. niger} in chemostat culture has been demonstrated \cite{jorgensen2007}, while Pollack and colleagues noted that the growth of a glucoamylase producing strain of \emph{A. oryzae} ceased when starved of glucose, before growth recommenced at a significantly reduced specific growth rate, fuelled by endogenous carbon \cite{pollack2008}. M\"{u}ller and colleagues noted a small decrease in the specific growth rate of both a wild-type strain of \emph{A. oryzae} and a mutant strain at a low glucose concentration \cite{muller2002}.

Variations in media composition have also been demonstrated as a successful means of inducing a change in the dynamics of hyphal development. The specific growth rate of \emph{A. niger} was higher in the presence of sucrose compared to glucose and fructose \cite{bizukojc2006}, while it has also been reported that the addition of kaolin had a positive effect \cite{ali2006}. McIntyre and colleagues described an influence of media composition on the maximum specific growth rate of \emph{Mucor circinelloides} (syn. \emph{racemosus}) \cite{mcintyre2002}, but the specific growth rate of \emph{M.~circinelloides} was later reported to be relatively independent of both carbon source type and concentration \cite{lubbehusen2003}. Meanwhile, the concentration of both carbon and nitrogen sources was found to have a significant impact on the growth kinetics of \emph{Mortierella alpine} \cite{eypark2002}.

\subsection{Aims of the work in this chapter}

Despite this significant progress in elucidating relationships between kinetic parameters of growth in filamentous microbes, the majority of the aforementioned studies have focussed on the submerged culture format. As such, there is currently a dearth of \lq micro-analyses' of filamentous microorganisms on solid substrates. The aims of the work presented in this chapter were to (a) examine the kinetic development of \emph{A. oryzae} on a solid substrate, using the system described in Chapters~\ref{ch:DevImagAnal} and \ref{ch:NitroAssay} (b) establish the basic kinetic parameters for this system and (c) investigate whether a change in media composition results in a change in kinetic parameters.