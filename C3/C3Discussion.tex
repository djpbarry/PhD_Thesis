\section{Discussion}

There is currently a dearth of techniques available for the study of filamentous microbial development on solid substrates, particularly for analysis of early-stage differentiation ($< 24$~hours post-inoculation). While reports of microscopic examination of fungi have been common in the literature \cite{cox1998}, the majority have focussed on the submerged culture format. Conventional techniques for the examination of filamentous microbes on solid culture, such as those involving \lq tape-lifts' \cite{harris2000,rodriguez-tudela1991}, require considerable skill and experience to master and often incur significant disruption to the fungal conformations. Given the potential for differential gene and protein expression in solid culture compared to the submerged format \cite{ishida1998,oda2006,tebiesebeke2002}, the ability to study the growth of filamentous microbes in detail in such a system would be highly advantageous and may lead to a greater understanding of the relationship between morphology and productivity.

In presenting complex fungal conformations in an essentially two-dimensional format, membrane immobilization confines cultures to a single focal plane, while maintaining the natural spatial arrangement of vegetative mycelia encountered in solid state culture. The membrane also presents a very low level of background or artifact (Fig.~\ref{fig:BackgroundDist}), which is highly advantageous for images that are to be analysed automatically, assuming the biomass present is uniformly stained. In such a scenario, the image histogram will be characterised by two well-separated peaks (bi-modal; Fig.~3.6) and object segmentation is easily achieved by calculating a single grey-level threshold. While eliminating artifact completely is obviously impossible, minimising the occurrence of small particles is desirable. The system developed in Chapter~\ref{ch:DevImagAnal} is capable of identifying objects that conform to a certain morphological description, but a small level of artifact will inevitably be incorrectly classified as hyphae and/or spores and included in the results. As such, measures to minimise the occurrence of artifact within the sample preparation should always be exploited. Due to the use of filtration in sample preparation, the possibility exists that sediment may accumulate on the membrane surface during processing and/or inoculation. It is therefore important to ensure that all solutions used are completely clarified so as to minimise the levels of artifact present in the resultant sample.

Perhaps the most significant advantage of this method of fungal examination is the ability to use high-power oil-immersion objective lenses, which permit the identification of fine details in the hyphae (Fig.~\ref{fig:MembranesHiMag} \& \ref{fig:CFWHiMag}).  The versatility of the assay is further emphasised by the demonstrated compatibility with fluorescent staining (Fig.~\ref{fig:Submergedb} \& \ref{fig:CFWHiMag}). Studies of filamentous microorganisms in solid culture using fluorescent stains have been quite rare, so such a system is of considerable potential benefit in, for example, quantifying active regions of hyphae in solid culture using calcofluor white staining. Furthermore, combining fluorescence microscopy with high-power magnification provides potential new perspectives on solid-state development.

Additional advantages of this system include the ability to modify the nutrient environment by simple membrane transfer. Such was employed by Ishida and colleagues in their investigation of gene expression by \emph{A. oryzae} \cite{ishida1998}. It is also well suited to investigating diffusion-mediated processes, such as antimicrobial susceptibility testing, or trophic responses to substrate gradients, for example. Other potential applications include investigations of germinative potential of populations of spores in response to various stimuli \cite{sautour2001}.

Membrane-immobilisation also permits the simultaneous examination of a relatively large number of hyphal elements, the size of the population being limited only by the area of the membrane. All elements on a single membrane are subjected to the same treatments during sample preparation, thus ensuring consistency within the population. Archiving of membrane-bound material is also possible, as the culture is fixed, stained, and killed during specimen preparation. It was found that samples stored for up to three years after preparation were still suitable for imaging.

The assay is, however, highly sensitive to the choice of membrane used as support (Fig.~\ref{fig:Membranes}). This is an important consideration as it has been demonstrated that membranes with similar product descriptions produced by different manufacturers can result in different growth patterns (not shown). In some cases, it has been found that even the orientation of the membrane can influence the morphology of the organism. Indeed, Jones and colleagues claimed that membrane orientation can affect both cellular attachment and stain uptake \cite{jones1994}, while a difference in membrane pore size was found to alter gene expression in recombinant \emph{A. oryzae} \cite{ishida1998}. Membrane-immobilisation also involves a considerable amount of time devoted to sample processing (approximately 3~hours). However, this compares very favourably with other reports of processing times for membrane-bound cultures of up to 24~hours, which also involved a much greater degree of subtle sample manipulation and sophisticated equipment \cite{jones1994}.

It has been reported that the use of membranes in SSF is not representative of growth in the absence of a support, with oxygen diffusion in particular being restricted and, as a result, growth and metabolite production being limited \cite{rahardjo2004}. However, the study in question examined growth from approximately 20~hours post-inoculation onwards, and significant deviation between the membrane-bound culture and the control were not observed until approximately 40~hours post-inoculation. While these findings should obviously be considered when membrane-immobilised cultures are a target of study, the aim in the development of the assay reported here was to provide, in particular, a means of analysing early-stage hyphal differentiation (0~-~24 hours). As such, concerns regarding oxygen limitations are somewhat unwarranted in this context.

Of further consideration is the use of lactophenol cotton blue (LPCB) as a means of enhancing contrast for the visualisation of fungal preparations. It has been noted that only young, recently developed hyphae tend to stain intensely when exposed to LPCB, such as those at the periphery of a colony, for example. Those hyphae that are more distal to the colony perimeter tend to remain completely unstained, or else take on a very pale blue appearance. This may be explained by a transformation in the composition of the cell wall as hyphae mature \cite{carlile2001}. As such, LPCB may only be suitable for staining actively-growing hyphae. Such an observation has been made in cultures of \emph{A. oryzae} stained with calcofluor white \cite{amanullah2002}. Considering the mechanisms of stain uptake to be similar for both compounds, this may infer that LPCB is also not suitable for staining older, \lq non-active' regions. However, such a phenomenon could prove useful in SSF, whereby the actively growing region at the colony periphery could be easily identified. The prevalence of \lq patchy' uptake of LPCB noted in this study on hyphae that are not dried prior to staining (Fig.~\ref{fig:WetDryStain}) calls into question the use of LPCB-stained wet-mounts and their suitability for image analysis \cite{amanullah2000,li2000,bizukojc2006,haack2006}.

The system has also been demonstrated as an effective means of examining fungal growth in submerged culture. By filtering samples through a membrane, rather than, for example, drying directly onto a microscope slide, any solutes present in the broth are removed, reducing the level of artifact in the subsequent field of view. This also results in the fungal conformations being presented in essentially two dimensions, facilitating an ease of analysis. While fungal biomass cultivated in submerged medium often has a significant 3-dimensional character, it is common for the sample to be \lq compressed' into two dimensions, generally by \lq wet-mounting' between a microscope slide and cover-slip \cite{amanullah2000,li2000,papagianni2002,bhargava2003}. Compared to pipetting samples onto a slide, decanting a sample into a filtration device also obviates any concern about selectively sampling objects below a certain size and also aids in the uniform distribution of biomass across the membrane surface. Membrane immobilisation also obviates the need for multiple slide preparations (depending on membrane area), thus ensuring further consistency. This may be particularly relevant in the event that calculations of biomass per unit volume are being derived.

Examination of the development of filamentous microbes on solid substrates has typically been of low resolution capability and the subsequent analysis may involve considerable human intervention \cite{larralde-corona1997}. However, membranes have previously been used in the assessment of fungal growth on solid substrates and have been combined with image processing and light microscopy, but generally with low magnification, low resolution capability. Cellulose acetate membranes have been used to study the growth of \emph{Trichoderma virens} \cite{cross2004} in conjunction with a dissecting microscope. Image analysis has been used in the enumeration of the fractal dimension of \emph{Pycnoporus cinnabarinus} \cite{jones1997} and \emph{Trichoderma viride} \cite{hitchcock1996} colonies immobilized on polycarbonate and cellophane membranes, respectively. The quantification of septation in \emph{Streptomyces tendae} was achieved using cellophane membranes in conjunction with image analysis \cite{reichl1990}, but this method involved the highly skilful transfer of mycelial matter from the membrane to a microscope slide. Polycarbonate membranes were also used in the immobilisation of \emph{Pycnoporus cinnabarinus} prior to histochemical laccase detection \cite{jones1999} and to limit growth to two dimensions in the localisation of protein secretion by \emph{A. niger} in solid culture \cite{wosten1991}.

\section{Conclusions}

There is renewed interest in the solid culture format, due in part to recent findings suggesting differential metabolite expression compared to submerged culture. While a variety of organisms have been studied in the literature in order to relate morphology to metabolite excretion, the vast majority of these studies have focussed on the submerged culture format. Those that have investigated the solid culture format have generally been of a low-resolution, low-magnification capability, or else are suited to the study of just a small population. There is thus a need to develop systems suitable for the study of filamentous microbes on solid substrates, particularly with a view to generating samples suitable for presentation to an image analysis system.

The use of inert membranes to restrict growth of filamentous microbes to two dimensions has been reported in the literature and their use was investigated here. The oil-treated membrane-immobilised culture presented a high-contrast, low-artifact field of view, suitable for imaging and automatic analysis. A low level of background artifact was encountered and, when dried prior to staining, hyphae exhibited high-contrast elements, easily segmentable from the background. The effect of drying of samples prior to staining was not significant.

Of particular significance is the provision of an ability to observe solid-cultured samples with high-power, oil-immersion objective lenses, permitting the capture of fine details. Such a capability may be of particular use in the study of hyphal morphogenesis. The utility of the method in the examination of samples taken from submerged culture was also demonstrated, reducing the complex conformations found in a typical fermentation from three down to two dimensions, facilitating an ease of imaging and subsequent analysis. A compatibility with fluorescent staining was also shown, with fine details such as active hyphal regions and septation illustrated.

The utility of the membrane-immobilisation system demonstrated here suggests that, in conjunction with image processing, such a culture format would prove useful for quantitative assessment of early filamentous microbial development. The examination of a large population of mycelia is facilitated, each of which is simultaneously subjected to the same treatment (ensuring sample consistency), resulting in high-contrast samples with low levels of artifact, suitable for accurate automated analysis. There currently exists a pressing need for such studies, given the lack of computational methods for solid-state fermentations.