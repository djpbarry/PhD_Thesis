\section{Introduction}

A large number of products of commercial significance are produced using filamentous microbes, such as enzymes, antibiotics and organic acids \cite{papagiannireview}. The majority of these compounds are produced in the submerged culture format (SmF), due to the greater ease with which such fermentations may be controlled and the metabolite of interest separated from biomass. However, recent reports suggest elevated metabolite expression by filamentous microbes when cultured in solid-state culture (SSF) compared to submerged \cite{tebiesebeke2002,oda2006}. Furthermore, SSF is not without merits, such as low energy requirements and cheap raw materials, which has led to the utilisation of this fermentation format in the production of a range of bioactive compounds, organic acids and industrial enzymes \cite{pandey2000}.

Key to fostering an understanding of metabolite expression in SSF is an ability to accurately and rapidly assess the spatial distribution of the organism. Relationships between productivity and gross morphology are well-established in many processes \cite{papagiannireview}, with some studies also suggesting links between metabolite expression and micro-morphological indicators, such as branch formation \cite{amanullah2002,muller2002}. However, the vast majority of these studies have been restricted to the SmF format and there is therefore a need to develop methods suitable for use in examining hyphal growth on solid culture. Conventional techniques for microscopically studying moulds, such as those employed in clinical mycology laboratories, include \lq tease-mounts', whereby a small amount of biomass is torn from a solid culture with needles and \lq wet-mounted' in a suitable stain (such as lactophenol cotton blue), or a variety of \lq tape-touch' methods \cite{harris2000,rodriguez-tudela1991}. These techniques require a considerable level of skill, often incur significant disruption to the fungal conformations and are typically suited to the isolation of characteristic reproductive structures for specific strain identification, rather than the examination of early hyphal differentiation.

\subsection{Assessment of two-dimensional microbial development}

One of the chief difficulties associated with studying the morphology of filamentous microorganisms is their significant three-dimensional character, even in solid culture, where the microbe may produce both aerial hyphae and penetrative structures that bore down into the substrate. Such three-dimensional arrangements complicate the visualisation of fungal conformations, as capturing morphological features in a single focal plane is often impossible, particularly at high magnification. One of the more successful means of managing this particular problem has been to cultivate the microbe of interest in a \lq flow-through cell', a small chamber designed specifically to restrict the growth of the organism to two dimensions in a controlled environment. Such devices have been successfully applied to the study of \emph{Mucor circinelloides} \cite{lubbehusen2003,lubbehusen2004}, \emph{Aspergillus oryzae} \cite{spohr1998,muller2002,pollack2008} and various \emph{Mortierella} species \cite{eypark2002,eypark2006}. While such studies have provided invaluable data on the development of individual hyphal elements, they only permit the simultaneous study of a very small population and, as such, derivation of global, species-specific statistics is difficult. Furthermore, such an apparatus is only suitable for the study of submerged growth.

In designing a means of visualising fungal hyphae, of principal concern is the provision of a field of view that is predominantly free of artifact, provides a high-magnification, high-resolution representation of the microbe and provides good contrast between the hyphae and background, particularly if the intention is for the resultant images to be analysed automatically. While the cultivation of filamentous microbes in SSF has received considerable attention in the literature, many of the methods utilised for the visualisation of hyphal development do not meet the above criteria. Many studies of SSF have been of a relatively low-magnification capability, with assessment of mycelia often limited to the macroscopic scale and morphological measures have typically been limited to \lq global' parameters, such as projected area of biomass \cite{kampichler2004,couri2006}, colony fractal dimension \cite{wells1998,kampichler2004} or colony expansion rate, derived from manual measures of colony diameter \cite{marin1998,kampichler2004,rahardjo2005b,golinski2008}. Other studies have focussed directly on quantification of metabolite production rather than on hyphal growth. Olsson reported a means of mapping the glucose and phosphorus uptake by \emph{Fusarium oxysporum} on agar \cite{olsson1994}, while Jones and colleagues focussed on evaluating dye biotransformation by \emph{Pycnoporus cinnabarinus} and \emph{Phanerochaete chrysosporium} \cite{jones1993b}. 

Where microscopic assessments have been made, efforts to restrict growth to two dimensions often involve the inoculation of cover-slips, microscope slides or Petri dishes with minimal volumes of media, which are subsequently viewed with bright-field optics \cite{obert1990,sautour2001,tebiesebeke2005,rahardjo2005b,kubo2007}. Some methods involve the fixing and staining of the cultured cover-slip prior to observation \cite{lin2004}. Many require the application of a cover-slip on top of the culture to be scrutinised \cite{lin2004,bartnicki-garcia2000,dieguez-uribeondo2005}, \lq compressing' the biomass, which inevitably results in disruption of the hyphae present. Nevertheless, this form of approach has yielded good quality, high-magnification images in some cases \cite{bartnicki-garcia2000,dieguez-uribeondo2005}, providing detailed information on hyphal extension. However, the population studied is typically very small and substantial manual intervention was required in the analysis of the images. In other cases, samples of agar are excised from an existing culture and observed using bright-field or fluorescence microscopy \cite{medwid1984,nonomura2003,pardo2005}. Microscopic examination on substrates such as leaves often resulted in significant levels of artifact, requiring the development of imaging routines specifically for the purpose of recognising certain artifactual features, such as stomata \cite{daniel1995}. More specialised techniques, such as confocal microscopy, have also been employed to visualise the penetration of hyphae into a solid substrate \cite{nopharatana2003}.

\subsection{Membrane-immobilisation of filamentous microbes}

An alternative approach involves the cultivation of filamentous microbes immobilised on an inert membrane, placed on top of a solid nutrient medium. Such an arrangement permits an ease of separation of biomass from the medium following a suitable period of incubation, which can then be processed as required prior to microscopic analysis. An additional advantage of such a system is the planar nature of the resultant fungal growth, as the presence of the membrane presents a physical barrier to hyphal extension into the substrate. Restricting growth to two-dimensions in such a manner facilitates ease of microscopic examination and subsequent imaging, as all biomass is predominantly presented in a single focal plane.

There are a limited number of studies in the literature involving the use of membranes in the culturing of filamentous microbes (Table~\ref{tab:MembraneLit}). In some cases, the cultured membranes were subjected to image processing \cite{hitchcock1996,cross2004}, but the resultant images were typically of a low resolution and often contained significant levels of artifact. In the case of the method described by Reichl and colleagues, the skilful transfer of \emph{S. tendae} mycelia from cellophane membranes to a microscope slide (by bringing the slide into contact with the membrane) was required prior to analysis with fluorescence microscopy, which may have incurred disruption to the hyphae \cite{reichl1990}. However, the ability of the membrane to restrict growth to two dimensions has been documented \cite{hitchcock1996,jones1997,cross2004}, although in some cases, this involved \lq sandwiching' the culture between two membranes \cite{sone1999,cross2004}, which may have incurred damage to the hyphae upon the destructive sampling of the microcosm. However, further potential application of such a system has been demonstrated in the production of high-magnification images of laccase localisation in \emph{Pycnoporus cinnabarinus} hyphae by treating the membrane with microscopy-immersion oil, permitting the use of oil-immersion objective lenses \cite{jones1999}. The advantages of membrane-immobilisation outlined in these studies suggest that such a system warrants further investigation. The potential clearly exists for the production of planar, two-dimensional hyphal structures that may be routinely examined using high-magnification oil-immersion microscope lenses.

\begin{table}[htbp]
	\centering
	\footnotesize
	\caption{Reports in the literature of membrane-immobilisation of filamentous microbes}
	\label{tab:MembraneLit}
	\begin{tabularx}{(\textwidth - 1cm)}{l P R c}
		\toprule
		Organism & Material & Report & Reference \\ \midrule
		Various & Cellophane & Growth kinetics of filamentous microbes determined by continuous monitoring of immobilised cultures & \cite{trinci1974}\\
		\emph{S. tendae} & Cellophane & Cultivated on membranes prior to transfer to microscopic slides for fluorescent examination  & \cite{reichl1990}\\
		\emph{A. niger} & Polycarbonate & Membranes utilised to restrict growth to a layer of agarose & \cite{wosten1991}\\
		\emph{T. viride} & Cellophane & Measures of local fractal dimension and branching complexity derived & \cite{hitchcock1996}\\
		\emph{P. cinnabarinus} & Polycarbonate & Fractal dimension of membrane-immobilised cultures related to phenol-oxidase yield in SmF & \cite{jones1997}\\
		\emph{P. cinnabarinus} & Cellophane \& polycarbonate & Laccase detected histochemically in hyphae by growing on membranes overlaid on agar & \cite{jones1999}\\
		\emph{N. crassa} & Cellophane & Cultivated between two sheets of cellophane in analysis of hyphal branching & \cite{sone1999}\\
		\emph{T. virens} & Cellulose acetate & Investigated effects of temperature on growth & \cite{cross2004}\\
		\bottomrule
	\end{tabularx}
\end{table}

\subsection{Aim of the work in this chapter}

The principal aim of the work described in this chapter was to develop a two-dimensional growth assay by immobilising fungal spores on cellulose nitrate membranes, using \emph{A. oryzae} as the model organism. The suitability of the assay for producing samples appropriate for the image analysis system described in Chapter~\ref{ch:DevImagAnal} was investigated. Such a sample should consist of a low level of background and artifact while yielding contiguous high-contrast staining of fungal hyphae. As nitrocellulose membranes are rendered transparent with microscopic immersion oil, the use of high-magnification oil objective lenses is possible. Such examinations were pursued with a view to identifying fine details in hyphal structure, such as septation.

Means of optimising the assay were investigated, concentrating specifically on minimising processing time, maximising the contrast between stained hyphae and sample background, while minimising any disruption to the hyphal architecture. Finally, the use of membrane-immobilisation as a means of analysing samples of biomass taken from submerged culture was investigated, using both the conventional lactophenol cotton blue stain and calcofluor white, with a view to identifying sub-cellular details and active regions in the hyphae.