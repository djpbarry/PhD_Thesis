\section{Materials \& methods}

For the initial \lq proof-of-principle' experimentation, nitrocellulose squares (approximately $4 \times 4$~mm) were aseptically cut from a proprietary membrane (Millipore HAWG 047 SO; \mic{0.45}) by use of a sterile scissors. Each nitrocellulose square was overlaid evenly onto the surface of a malt agar plate and then inoculated with \ul{5} of a standardized conidial suspension (\inoc{1}{6}; Fig.~\ref{fig:NCDiag}). The plates were incubated at \celc{25} and test squares removed after various time intervals for drying, staining, and microscopic visualization. Membranes were dried in sterile Petri dishes (3~h, \celc{25} air incubator) and stained with lactophenol cotton blue (BBL Diagnostic Systems PL7054). The membranes were rinsed in PBS-Tween 80, re-dried at \celc{25} (3~h), and placed on microscope slides; a minimum volume of immersion oil was added (Olympus AX 9602), and images were captured at various magnifications (stated in figure legends). The optimised assay is described in Section~\ref{sec:OptAssay}.

\begin{figure}[tb]
	\centering
	\fbox{\pstool[width=10.5cm]{../C3/NCDiag}{
		\psfrag{A}[Bl]{\al Agar}
		\psfrag{S}[Bl]{\al Spore Suspension}
		\psfrag{N}[Bl]{\al Membrane}}}
	\caption{Schematic representation of inoculation of membrane on agar with spore suspension.}
	\label{fig:NCDiag}
\end{figure}

\subsection{Assessment of image background}

Ten regions of background ($320 \times 240$ pixels), defined as an area within an image devoid of any biomass, were manually selected in ten randomly selected images of membrane-immobilised hyphae. The green and blue bands of the RGB images were discarded and a 256-bin luminance histogram computed for the resultant 8-bit greyscale images. Each distribution was then aligned and plotted.